\subsection{Permutations}

\begin{slide}
    \begin{definition}[permutation]
        \vspace{0pt}
        \textbf{Permutation} of $\Omega$ is bijection $g : \Omega \to \Omega$. \\
        \textbf{Symmetric group} $\Sym(\Omega)$ is set of permutations of $\Omega$. \\
        (For $\Omega = [n] := \{1,\dotsc,n\}$, write $\Sym(n)$.)
    \end{definition}

    \onslide<2->{Write $1 = ()$ for identity. Write $i^g$ not $g(i)$ for \textit{image}.

        \onslide<3->{\textit{Cycle notation:} $g = (1,4,5)(2,6) \in \Sym(6)$ is:

            \begin{center}
                \begin{tikzpicture}[x=1cm,y=2cm]
                    \GraphInit[vstyle=Normal]
                    \tikzset{VertexStyle/.style={draw=none}}
                    \tikzset{EdgeStyle/.style={->}}
                    \Vertex{1} \EA(1){2} \EA(2){3} \EA(3){4} \EA(4){5} \EA(5){6}
                    \SO[L=1](1){1'} \SO[L=2](2){2'} \SO[L=3](3){3'} \SO[L=4](4){4'} \SO[L=5](5){5'} \SO[L=6](6){6'}
                    \onslide<4->{\Edge(1.south)(4'.north)}
                    \onslide<5->{\Edge(4.south)(5'.north)}
                    \onslide<6->{\Edge(5.south)(1'.north)}
                    \onslide<7->{\Edge(2.south)(6'.north)}
                    \onslide<8->{\Edge(6.south)(2'.north)}
                    \onslide<9->{\Edge(3.south)(3'.north)}
                    \tkzDefPoint(-0.5,-0.5){lab}\tkzLabelPoint[left](lab){$g$}
                \end{tikzpicture}
            \end{center}

            It means $\onslide<4->{1^g = 4,\ \onslide<5->{4^g = 5,\ \onslide<6->{5^g = 1,\ \onslide<7->{2^g = 6,\ \onslide<8->{6^g = 2,\ \onslide<9->{3^g = 3.}}}}}}$}}
\end{slide}

\begin{slide}
    \textit{Product/composition:} for $g,h \in \Sym(\Omega)$, $gh$ means ``first $g$, then $h$'', so $\alpha^{gh} = (\alpha^g)^h$. \onslide<2->{E.g. $g = (1,2,3) \in \Sym(4)$, \onslide<3->{$h = (1,3)(2,4) \in \Sym(4)$,}

    \begin{center}
        \begin{tikzpicture}[x=1cm,y=1.5cm]
            \GraphInit[vstyle=Normal]
            \tikzset{VertexStyle/.style={draw=none}}
            \tikzset{EdgeStyle/.style={->}}
            \Vertex{1} \EA(1){2} \EA(2){3} \EA(3){4}
            \SO[L=1](1){1'} \SO[L=2](2){2'} \SO[L=3](3){3'} \SO[L=4](4){4'}
            \Edge(1.south)(2'.north) \Edge(2.south)(3'.north) \Edge(3.south)(1'.north) \Edge(4.south)(4'.north)
            \tkzDefPoint(-0.5,-0.5){lab}\tkzLabelPoint[left](lab){$\sigma$}
            \onslide<3->{\SO[L=1](1'){1''} \SO[L=2](2'){2''} \SO[L=3](3'){3''} \SO[L=4](4'){4''}
                \Edge(1'.south)(3''.north) \Edge(3'.south)(1''.north) \Edge(2'.south)(4''.north) \Edge(4'.south)(2''.north)
                \tkzDefPoint(-0.5,-1.5){lab2}\tkzLabelPoint[left](lab2){$\tau$}
            }
            \onslide<4->{
                \EA[unit=3,L=1](4){c1} \EA[L=2](c1){c2} \EA[L=3](c2){c3} \EA[L=4](c3){c4}
                \EA[unit=3,L=1](4''){c1'} \EA[L=2](c1'){c2'} \EA[L=3](c2'){c3'} \EA[L=4](c3'){c4'}
                \Edge(c1.south)(c4'.north)
                \Edge(c4.south)(c2'.north)
                \Edge(c2.south)(c1'.north)
                \Edge(c3.south)(c3'.north)
                \tkzDefPoint(5.5,-1){lab3}\tkzLabelPoint[left](lab3){$\sigma\tau$}
            }
        \end{tikzpicture}
    \end{center}
    \onslide<4->{$$gh = (1,2,3)(1,3)(2,4) = (1,4,2) \in \Sym(4).$$

        \onslide<5->{\textit{Note:} here, $gh \neq hg$, since $1^{gh} = 4$ but $1^{hg} = (1^h)^g = 3^g = 1$. Identity $1 = ()$ satisfies $1 g = g 1 = g$ for $g \in \Sym(\Omega)$.}}}
\end{slide}

\subsection{Permutation groups}

\begin{slide}
    \begin{definition}[permutation group]
        \vspace{0pt}
        \textbf{Perm group} on $\Omega$ (of deg $n$) is subset $G \leq \Sym(\Omega)$ ($|\Omega| = n$) s.t.
        \begin{enumerate}[(i)]
            \item \textbf{(closure)} $gh \in G$ for $g,h \in G$; \pause
            \item \textbf{(identity)} $1 = () \in G$; \pause
            \item \textbf{(inverses)} $g^{-1} \in G$ for $g \in G$.
        \end{enumerate}
    \end{definition} \pause

    \begin{definition}[generator]
        \vspace{0pt}
        Set $X$ \textbf{generates} $G$ if every $g \in G$ is $g = x_1^{\pm 1} \dotsb x_r^{\pm 1}$ for some $r \in \N$, $x_i \in X$ \textbf{generators}; write $G = \langle X \rangle$.
    \end{definition} \pause

    \begin{example}[dihedral group]
        \vspace{0pt}
        Let $r = (1,2,3,4),s = (1,4)(2,3) \in \Sym(4)$. \textbf{Dihedral group} is $D_8 := \langle r,s \rangle = \{1,r,r^2,r^3,s,sr,sr^2,sr^3\}$, \textit{``symmetries of square''}.
    \end{example}
\end{slide}

% \begin{slide}
%     \begin{definition}[permutation group]
%         \vspace{0pt}
%         \textbf{Perm group} on $\Omega$ (of deg $n$) is subset $G \leq \Sym(\Omega)$ ($|\Omega| = n$) s.t.
%         \begin{enumerate}[(i)]
%             \item \textbf{(closure)} $gh \in G$ for $g,h \in G$; \pause
%             \item \textbf{(identity)} $1 = () \in G$; \pause
%             \item \textbf{(inverses)} $g^{-1} \in G$ for $g \in G$.
%         \end{enumerate}
%     \end{definition} \pause

%     \begin{example}[alternating group]
%         \vspace{0pt}
%         \textbf{Alternating group} $\Alt(3) = \{(),(1,2,3),(1,3,2)\} < \Sym(3)$. \pause \\
%         In general, $\Alt(n)$ is all \textit{even} permutations of $[n]$ (product of even \# of \textit{transpositions} $(i,j)$, e.g. $(1,2,3) = (1,2)(1,3) \in \Sym(n)$).
%     \end{example}
% \end{slide}

% \subsection{Generating a group}

% \begin{slide}
%     \begin{definition}[generator]
%         \vspace{0pt}
%         Set $X$ \textbf{generates} $G$ if every $g \in G$ is $g = x_1^{\pm 1} \dotsb x_r^{\pm 1}$ for some $r \in \N$, $x_i \in X$ \textbf{generators}; write $G = \langle X \rangle$. \pause

%         (If $G = \langle X \rangle$ for some $X$ with $|X| = 1$, $G$ is \textbf{cyclic}.)
%     \end{definition} \pause

%     \begin{example}[cyclic group]
%         \vspace{0pt}
%         Consider $\Alt(3) = \{(),(1,2,3),(1,3,2)\}$: $(1,2,3)^2 = (1,3,2)$, $(1,2,3)^3 = ()$, so $\Alt(3) = \langle(1,2,3)\rangle$ is cyclic (only for $n = 3$).
%     \end{example} \pause
% \end{slide}

\subsection{Group actions}

\begin{slide}
    \begin{definition}[group action]
        \vspace{0pt}
        For (perm) group $G$ and set $\Omega \neq \emptyset$, a \textbf{$G$-action} is map $\Omega \times G \to \Omega$, $(\alpha,g) \mapsto \alpha^g$ s.t. $\alpha^1 = \alpha$ and $\alpha^{gh} = (\alpha^g)^h$ for $\alpha \in \Omega$ and $g,h \in G$. \textbf{Degree} of action is $|\Omega|$.
    \end{definition}

    \textit{Idea:} $\alpha \in \Omega$ is \textit{state}, apply \textit{move} $g \in G$ to get state $\alpha^g \in \Omega$, in way that respects permutation product. \pause

    \begin{example}[natural action]
        \vspace{0pt}
        $G \leq \Sym(\Omega)$ acts on $\Omega$ by $\alpha^g := \alpha^g$ (image) for $\alpha \in \Omega$, $g \in G$.
    \end{example} \pause

    % $G$-action on $\Omega$ yields \textit{homomorphism}
    % $$\varphi : G \to \Sym(\Omega), \quad g \mapsto \varphi(g) = (\alpha \mapsto \alpha^g) \in \Sym(\Omega);$$
    % image is perm group on $\Omega$ with natural action. \\
    % \textit{Hom:} $\varphi(gh) = (\alpha \mapsto \alpha^{gh}) = (\alpha \mapsto \alpha^g)(\alpha \mapsto \alpha^h) = \varphi(g)\varphi(h)$.
\end{slide}

\begin{slide}
    \begin{example}[dihedral group]
        \vspace{0pt}
        Recall $D_8 = \langle r,s \rangle = \{1,r,r^2,r^3,s,sr,sr^2,sr^3\} \leq \Sym(4)$.

        \onslide<2->{\textit{Note:} $r = (1,2,3,4)$, $s = (1,4)(2,3)$, $sr = (2,4)$. Action of $D_8$ on vertices of square (labelled by $[4]$): $g \in D_8$ sends vertex at $i$ to $i^g$.}

        \begin{center}
            \begin{tikzpicture}[x=1cm,y=1cm]
                \GraphInit[vstyle=Classic]
                \tikzset{VertexStyle/.append style={minimum size=1pt}}
                % TOP LEFT
                \tkzDefPoint(0.5,0.25){l2}
                \tkzDefPoint(0.5,-1.25){l3}
                \tkzDefPoint(-0.25,0.25){l4}
                \tkzDefPoint(1.25,-1.25){l5}
                \tikzset{EdgeStyle/.style={dotted}}
                \onslide<4->{\Edge(l2)(l3)} % s
                \onslide<6->{\Edge[color=metrogreen](l4)(l5)} % sr
                \tikzset{EdgeStyle/.style={-}}
                \onslide<3->{{\Vertex[Lpos=180]{1} \SO[Lpos=180](1){2} \EA(2){3} \NO(3){4}
                            \AddVertexColor{red}{1}
                            \AddVertexColor{green}{2}
                            \AddVertexColor{blue}{3}
                            \AddVertexColor{yellow}{4}}
                    \Edges(1,2,3,4,1)
                    \tikzset{EdgeStyle/.style={->}}
                    \tkzDefPoint(1.5,-0.5){a1}
                    \tkzDefPoint(3.5,-0.5){a2}
                    % TOP RIGHT
                    \tikzset{EdgeStyle/.style={->}}
                    \tkzDefPoint(0.5,-1.5){a1}
                    \tkzDefPoint(0.5,-2.5){a2}
                    \onslide<4->{\Edge[label=$s$,labelcolor=metropolis](a1)(a2)}
                    % BOTTOM LEFT
                    \tikzset{EdgeStyle/.style={-}}
                    \onslide<4->{\Vertex[x=0,y=-3,Lpos=180]{1} \SO[Lpos=180](1){2} \EA(2){3} \NO(3){4}
                        \AddVertexColor{yellow}{1}
                        \AddVertexColor{blue}{2}
                        \AddVertexColor{green}{3}
                        \AddVertexColor{red}{4}
                        \Edges(1,2,3,4,1)}
                    \tikzset{EdgeStyle/.style={->}}
                    \tkzDefPoint(1.5,-3.5){a1}
                    \tkzDefPoint(3.5,-3.5){a2}
                    \onslide<5->{\Edge[label=$r$,labelcolor=metropolis](a1)(a2)
                        \tkzDefPoint(2.5,-3){l1}\tkzLabelPoint[centered](l1){$\curvearrowleft$}}
                    % BOTTOM RIGHT
                    \tikzset{EdgeStyle/.style={-}}
                    \onslide<5->{\Vertex[x=4,y=-3,Lpos=180]{1} \SO[Lpos=180](1){2} \EA(2){3} \NO(3){4}
                        \AddVertexColor{red}{1}
                        \AddVertexColor{yellow}{2}
                        \AddVertexColor{blue}{3}
                        \AddVertexColor{green}{4}
                        \Edges(1,2,3,4,1)}
                    \tikzset{EdgeStyle/.style={->}}
                    \tkzDefPoint(1.5,-1.5){a1}
                    \tkzDefPoint(3.5,-2.5){a2}
                    \onslide<6->{\Edge[label=$sr$,labelcolor=metropolis,labeltext=metrogreen,color=metrogreen](a1)(a2)}}
            \end{tikzpicture}
        \end{center}
    \end{example}
\end{slide}

\subsection{Orbits and stabilisers}

\begin{slide}
    \begin{definition}[orbit]
        \vspace{0pt}
        If $G$ acts on $\Omega$, then \textbf{orbit} of $\alpha \in \Omega$ is $\alpha^G := \{\alpha^g : g \in G\}$. \\
        \textit{Idea:} states $\alpha^g \in \Omega$ reachable from fixed $\alpha \in \Omega$ by moves $g \in G$.
    \end{definition}

    One orbit only: \textbf{transitive} action. \pause

    \begin{definition}[stabiliser]
        \vspace{0pt}
        If $G$ acts on $\Omega$, then \textbf{stabiliser} of $\alpha \in \Omega$ is $G_\alpha := \{g \in G : \alpha^g = \alpha\}$. \\
        \textit{Idea:} moves $g \in G$ that fix given $\alpha \in \Omega$. \pause
    \end{definition}
\end{slide}

\begin{slide}
    Orbit $\alpha^G$: states $\alpha^g \in \Omega$ reachable from fixed $\alpha$ by moves $g \in G$. \\
    Stabiliser $G_\alpha$: moves $g \in G$ that fix given $\alpha$.

    \begin{example}[dihedral group]
        \vspace{0pt}
        Recall $G = D_8 = \langle r,s \rangle = \{1,r,r^2,r^3,s,sr,sr^2,sr^3\} \leq \Sym(4)$ where $r = (1,2,3,4)$, $s = (1,4)(2,3)$.

        Orbit of 1: $1^1 = 1$, $1^r = 2$, $1^{r^2} = 3$, $1^{r^3} = 4$, so $1^G = [4]$ (transitive). \pause

        Stabiliser of 1: $sr = (2,4)$, $sr^2 = (1,2)(3,4)$, $sr^3 = (1,3)$, so $G_1 = \{(),(2,4)\} = \{1,sr\}$. \pause

        \textit{Note:} $|1^G||G_1| = 4 \cdot 2 = 8 = |G|$. Coincidence?
    \end{example} \pause

    \begin{theorem}[orbit-stabiliser]
        \vspace{0pt}
        If $G$ acts on $\Omega$, then for $\alpha \in \Omega$, $|\alpha^G||G_\alpha| = |G|$.
    \end{theorem}
\end{slide}

\subsection{Blocks and primitivity}

\begin{slide}
    \begin{definition}[block]
        \vspace{0pt}
        If $G$ acts transitively on $\Omega$ and $\Delta \subseteq \Omega$, let $\Delta^g := \{\alpha^g : \alpha \in \Delta\}$. \\
        A \textbf{block} is $\Delta \subseteq \Omega$ with $\Delta^g = \Delta$ or $\Delta^g \cap \Delta = \emptyset$ for all $g \in G$. \pause

        Block is \textbf{nontrivial} if $|\Delta| > 1$ and $\Delta \neq \Omega$.
    \end{definition}

    \textit{Examples of blocks:} singletons, $\Omega$, orbits. \pause

    \begin{definition}[primitivity]
        \vspace{0pt}
        A \textit{transitive} $G$-action is \textbf{primitive} if there are no nontrivial blocks; otherwise it is \textbf{imprimitive}.

        If $G$ is perm group with primitive natural action, $G$ is \textbf{primitive}.
    \end{definition} \pause

    For block $\Delta$, define \textbf{block system} $\Sigma = \{\Delta^g : g \in G\}$ (partitions $\Omega$); then $G$ acts on $\Sigma$; if $\Delta$ is \textit{maximal}, then acts primitively.
\end{slide}

\begin{slide}
    \begin{example}[dihedral group]
        \vspace{0pt}
        Recall $G = D_8 = \langle r,s \rangle = \{1,r,r^2,r^3,s,sr,sr^2,sr^3\} \leq \Sym(4)$ where $r = (1,2,3,4)$, $s = (1,4)(2,3)$, $sr = (2,4)$.

        Block is $\Delta = \{1,3\}$ (nontrivial) with block system $\Sigma = \{\{1,3\},\{2,4\}\}$ (opposite vertices stay opposite):
        \begin{center}
            \begin{tikzpicture}[x=1cm,y=1cm]
                \GraphInit[vstyle=Classic]
                \tikzset{VertexStyle/.append style={minimum size=1pt}}
                % TOP LEFT
                \tkzDefPoint(0.5,0.25){l2}
                \tkzDefPoint(0.5,-1.25){l3}
                \tkzDefPoint(-0.25,0.25){l4}
                \tkzDefPoint(1.25,-1.25){l5}
                \tikzset{EdgeStyle/.style={dotted}}
                \tikzset{EdgeStyle/.style={-}}
                {\Vertex[Lpos=180]{1} \SO[Lpos=180](1){2} \EA(2){3} \NO(3){4}
                    \AddVertexColor{red}{1}
                    \AddVertexColor{green}{2}
                    \AddVertexColor{red}{3}
                    \AddVertexColor{green}{4}}
                \Edges(1,2,3,4,1)
                \tikzset{EdgeStyle/.style={->}}
                \tkzDefPoint(1.5,-0.5){a1}
                \tkzDefPoint(3.5,-0.5){a2}
                \Edge[label=$r$,labelcolor=metropolis](a1)(a2)
                \tkzDefPoint(2.5,0){l1}\tkzLabelPoint[centered](l1){$\curvearrowleft$}
                % TOP RIGHT
                \tkzDefPoint(3.75,-0.5){l2}
                \tkzDefPoint(5.25,-0.5){l3}
                \tikzset{EdgeStyle/.style={dotted}}
                \tikzset{EdgeStyle/.style={-}}
                \Vertex[x=4,y=0,Lpos=180]{1} \SO[Lpos=180](1){2} \EA(2){3} \NO(3){4}
                \AddVertexColor{green}{1}
                \AddVertexColor{red}{2}
                \AddVertexColor{green}{3}
                \AddVertexColor{red}{4}
                \Edges(1,2,3,4,1)
            \end{tikzpicture}
        \end{center}
        e.g. $\Delta^r = \{2,4\}$, \pause $\Delta^s = \{4,2\}$, $\Delta^{sr} = \{1,3\} = \Delta$.

        $D_8$ acts imprimitively on $[4]$ but primitively on $\Sigma$ (degree 2).
    \end{example}
\end{slide}