\subsection{Bases and stabiliser chains}

\begin{slide}
    \begin{definition}[Base, stabiliser chain]
        \vspace{0pt}
        If $G \leq \Sym(\Omega)$, distinct elts $B = [\beta_1,\dotsc,\beta_r] \subseteq \Omega$ is \textbf{base} for $G$ if $G_{\beta_1,\dotsc,\beta_r} = 1$. (\textit{Recall:} $G_{\beta_1,\dotsc,\beta_r} = \{g \in G : \beta_1^g = \beta_1,\dotsc,\beta_r^g = \beta_r\}$.) \pause

        Corresponding \textbf{stabiliser chain} is
        \[G = G^0 \geq G^1 \geq \dotsb \geq G^r = 1\]
        where $G^i = G^{i-1}_{\beta_i} = G_{\beta_1,\dotsc,\beta_i}$.
    \end{definition} \pause

    Base $B$ contains elts of $\Omega$ such that only $1 \in G$ fixes every $\beta_i \in B$. (Short base desirable: how to compute \textbf{min base} of length $b(G)$?) \pause

    \begin{theorem}[Blaha, 1992]
        \vspace{0pt}
        Problem of finding minimum base for $G$ is NP-complete (if $\mathrm{P} \neq \mathrm{NP}$, then no polynomial time algorithm).
    \end{theorem}
\end{slide}

\begin{slide}
    \begin{example}[Rubik's group]
        \vspace{0pt}
        Using \texttt{BaseOfGroup} cmd in \texttt{GAP}, base of $\mathcal{G}$ of size 18 is
        % \lstinputlisting{code/rubiks_group_base.gap} \pause
        $$B = [ 1, 3, 6, 8, 2, 4, 5, 7, 12, 13, 14, 15, 16, 21, 23, 24, 29, 31 ].$$ \pause
        Contains: 7 corner stickers (from 7 of 8 corners), 11 edge stickers (from 11 of 12 edges).
    \end{example} \pause

    \begin{theorem}
        \vspace{0pt}
        For Rubik's group $\RC$, $b(\RC) = 18$.
    \end{theorem}
\end{slide}

\begin{slide}
    Stabiliser chain implemented in \texttt{GAP}; useful in algorithms. \pause

    Let $G = \langle X \rangle \leq \Sym(\Omega)$ have base $B$ and stabiliser chain
    $$G = G^0 \geq G^1 \geq \dotsb \geq G^r = 1.$$ \pause

    \begin{alertblock}{Problem (random element generation)}
        \vspace{0pt}
        Generate uniformly random element of $G$.

        (\textit{Alternative:} \pause random product of generators in $X$; transitive Markov chain on Cayley graph (find mixing time?); distribution?)
    \end{alertblock} \pause

    \begin{alertblock}{Problem (membership testing)}
        \vspace{0pt}
        For $g \in \Sym(\Omega)$, test if $g \in G$.

        (\textit{Application}: \pause check if restickering of Rubik's cube is valid state.)
    \end{alertblock}
\end{slide}

\subsection{What is the size of the Rubik's group?}

\begin{slide}
    \begin{theorem}[size of perm group]
        \vspace{0pt}
        If $B = [\beta_1,\dotsc,\beta_r]$ is base for $G \leq \Sym(\Omega)$ with stabiliser chain $G = G^0 \geq G^1 \geq \dotsb \geq G^r = 1$, then
        $$|G| = |\beta_1^{G^0}||\beta_2^{G^1}| \dotsb |\beta_r^{G^{r-1}}|.$$
    \end{theorem}

    \onslide<2->
    Orbits and stabilisers can be easily computed (e.g. using \texttt{GAP}).

    \onslide<3->{Implementing base and stabiliser chain for Rubik's group $\RC$ (using \texttt{BaseOfGroup} and \texttt{StabChain} cmds), \texttt{GAP} computes:}

    \onslide<4->{
        \begin{corollary}
            \vspace{0pt}
            $|\RC| = 43\,252\,003\,274\,489\,856\,000 \approx 4.3 \cdot 10^{19}$.
        \end{corollary}

        % (\textit{Note:} $|\RC| = 2^{27} \cdot 3^{14} \cdot 5^3 \cdot 7^2 \cdot 11$. Thus no move of order 13.)
    }
\end{slide}