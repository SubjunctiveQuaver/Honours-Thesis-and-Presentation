% \documentclass{beamer}
\documentclass[handout]{beamer}

% Settings and commands
\usepackage[utf8]{inputenc} % Accept different input encodings
\usepackage{mathtools} % Extra symbols and fixes to amsmath
\usepackage{thmtools} % Theorem tools
\usepackage{graphicx}
% \usepackage[shortlabels]{enumitem} % Customising lists - incompatible with beamer
\usepackage{centernot} % Centred negation slash
\usepackage{hyperref}
\usepackage{tcolorbox} % Boxes surrounding text
\usepackage{tikz-cd} % Commutative diagrams
% \usepackage{newtx} % Nicer font
\usepackage{libertine}
\usepackage[libertine]{newtxmath}
\usepackage{bookmark} % Hyperlinks and bookmarks
\usepackage{inconsolata}
\usepackage{xcolor}
\usepackage{tkz-base}
\usepackage{tkz-graph}
\usepackage{tkz-euclide}
\usepackage{tikzscale}
\usepackage{listings}
\usepackage{csquotes}
\usepackage{accsupp}
\newcommand{\emptyaccsupp}[1]{\BeginAccSupp{ActualText={}}#1\EndAccSupp{}}

\usetheme[progressbar=frametitle]{metropolis}
\usefonttheme{professionalfonts}
\newcounter{insub}
\AtBeginSubsection[]{\setcounter{insub}{0}}
\AtBeginEnvironment{slide}{\addtocounter{insub}{1}}
\newenvironment<>{slide}{\begin{frame}#1%
        \frametitle{\subsecname{}\ifthenelse{\theinsub=1}{}{\enskip(\roman{insub})}}
        }{\end{frame}}
% \setbeamertemplate{theorems}[numbered]
\definecolor{metropolis}{HTML}{FAFAFA}
\definecolor{metrogreen}{HTML}{14B03D}
\definecolor{metrorange}{HTML}{EB811B}

% ---------------------------------------- %
% SETTINGS

\tcbuselibrary{breakable}
\newtcolorbox{framedbox}{standard jigsaw, opacityback=0, boxrule=1pt, breakable, before skip=8pt, after skip=8pt}

% \let \originalLeft \left % Fixing \left and \right
% \let \originalRight \right
% \renewcommand{\left}{\mathopen{} \mathclose \bgroup \originalLeft}
% \renewcommand{\right}{\aftergroup \egroup \originalRight}

\newtheorem{proposition}{Proposition}

%% tkz-graph
% \GraphInit[vstyle=Hasse]
% \GraphInit[vstyle=Welsh]
% \GraphInit[vstyle=Normal]
% \GraphInit[vstyle=Classic]
\SetGraphUnit{1}
\SetVertexMath
\tikzset{VertexStyle/.append style={minimum size=8pt}}

%% listings
% Code block styles (listings)
\definecolor{codegreen}{rgb}{0,0.6,0}
\definecolor{codegray}{rgb}{0.5,0.5,0.5}
\definecolor{codepurple}{rgb}{0.58,0,0.82}
\definecolor{backcolour}{rgb}{0.95,0.95,0.92}
\lstdefinestyle{default}{
    basewidth={.5em,0.5em},
    basicstyle=\ttfamily,
    breakatwhitespace=false,
    breaklines=true,
    captionpos=b,
    xleftmargin=15pt,
    framexleftmargin=2pt,
    framexrightmargin=2pt,
    keepspaces=true,
    numbers=left,
    numbersep=5pt,
    tabsize=4,
    columns=fullflexible,
    backgroundcolor=\color{backcolour},
    commentstyle=\itshape\color{codegreen},
    keywordstyle=\color{magenta},
    numberstyle=\footnotesize\itshape\color{black}\emptyaccsupp,
    stringstyle=\color{codepurple}
}
% \lstdefinelanguage{GAP}{%
%     morekeywords={%
%             Assert,Info,IsBound,QUIT,%
%             TryNextMethod,Unbind,and,break,%
%             continue,do,elif,%
%             else,end,false,fi,for,%
%             function,if,in,local,%
%             mod,not,od,or,%
%             quit,rec,repeat,return,%
%             then,true,until,while%
%         },%
%     sensitive,%
%     morecomment=[l]\#,%
%     morestring=[b]",%
%     morestring=[b]',%
% }[keywords,comments,strings]
\lstset{style=default,language=GAP}
\input{commands.tex}

\title{\textbf{Minimum bases in permutation groups}}
\author{\textbf{Lawrence Chen}}
\institute{\textbf{Honours presentation}\\\textbf{Monash University}\\Supervised by A/Prof. Heiko Dietrich and\\Dr Santiago Barrera Acevedo}
\date{\today}

\titlegraphic{
\vspace{3.75cm}\flushright\includegraphics[width=0.4\textwidth]{graphics/rubiks_cube_title.png}
}

\setbeamerfont{subsection in toc}{size=\footnotesize}
\newcommand{\RC}{\mathcal{G}}
\newcommand{\RS}{\mathcal{S}}
% \metroset{block=fill}

\begin{document}

\begin{frame}[plain, noframenumbering]
    \titlepage
\end{frame}

\begin{frame}[plain, noframenumbering]{Contents}
    % \tableofcontents[hideallsubsections]
    \begin{columns}[t]
        \begin{column}{.5\textwidth}
            \tableofcontents[sections={1-2}]
        \end{column}
        \begin{column}{.5\textwidth}
            \tableofcontents[sections={3-4}]
        \end{column}
    \end{columns}

    \vspace{0.5cm} \pause
    \textit{Aim:} analyse Blaha's 1992 paper on NP-completeness of min base problem, and recent results for primitive perm groups.
\end{frame}

\subsection{Motivation: understanding the Rubik's cube}

\begin{slide}
    \begin{itemize}
        \item How can we represent \textit{operations} of a cube? \pause
        \item \textit{How many} states does a Rubik's cube have? \pause
        \item How can we better \textit{understand} operations of a cube? \pause
    \end{itemize}

    \textit{One answer:} using permutations and \textit{computational group theory}! \pause

    \begin{alertblock}{(J. A. Paulos, Innumeracy)}
        \vspace{4pt}
        \begin{quotation}
            Ideal Toy Company stated on the package of the original Rubik cube that there were \textbf{more than three billion} possible states the cube could attain. It's analogous to McDonald's proudly announcing that they've sold \textbf{more than 120} hamburgers.
        \end{quotation}
    \end{alertblock}
\end{slide}

\section{Some basic group theory}

\subsection{Permutations}

\begin{slide}
    \begin{definition}[permutation]
        \vspace{0pt}
        \textbf{Permutation} of $[n] := \{1,\dotsc,n\}$ is bijection $\sigma : [n] \to [n]$. \textbf{Symmetric group} $\Sym(n)$ is set of permutations of $[n]$.
    \end{definition}

    \onslide<2->{Write $1 = ()$ for identity. Write $i^\sigma$ not $\sigma(i)$ for \textit{image}.

        \onslide<3->{\textit{Cycle notation:} $\sigma = (1,4,5)(2,6) \in \Sym(6)$ is:

            \begin{center}
                \begin{tikzpicture}[x=1cm,y=2cm]
                    \GraphInit[vstyle=Normal]
                    \tikzset{VertexStyle/.style={draw=none}}
                    \tikzset{EdgeStyle/.style={->}}
                    \Vertex{1} \EA(1){2} \EA(2){3} \EA(3){4} \EA(4){5} \EA(5){6}
                    \SO[L=1](1){1'} \SO[L=2](2){2'} \SO[L=3](3){3'} \SO[L=4](4){4'} \SO[L=5](5){5'} \SO[L=6](6){6'}
                    \onslide<4->{\Edge(1.south)(4'.north)}
                    \onslide<5->{\Edge(4.south)(5'.north)}
                    \onslide<6->{\Edge(5.south)(1'.north)}
                    \onslide<7->{\Edge(2.south)(6'.north)}
                    \onslide<8->{\Edge(6.south)(2'.north)}
                    \onslide<9->{\Edge(3.south)(3'.north)}
                    \tkzDefPoint(-0.5,-0.5){lab}\tkzLabelPoint[left](lab){$\sigma$}
                \end{tikzpicture}
            \end{center}

            It means
            $$\onslide<4->{1^\sigma = 4,\ \onslide<5->{4^\sigma = 5,\ \onslide<6->{5^\sigma = 1,\ \onslide<7->{2^\sigma = 6,\ \onslide<8->{6^\sigma = 2,\ \onslide<9->{3^\sigma = 3.}}}}}}$$}}
\end{slide}

\begin{slide}
    \textit{Product/composition:} for $\sigma,\tau \in \Sym(n)$, $\sigma\tau$ means ``first $\sigma$, then $\tau$'', so $i^{\sigma\tau} = (i^\sigma)^\tau$. \onslide<2->{E.g. $\sigma = (1,2,3),\onslide<3->{\tau = (1,3)(2,4) \in \Sym(4)$,}

    \begin{center}
        \begin{tikzpicture}[x=1cm,y=1.5cm]
            \GraphInit[vstyle=Normal]
            \tikzset{VertexStyle/.style={draw=none}}
            \tikzset{EdgeStyle/.style={->}}
            \Vertex{1} \EA(1){2} \EA(2){3} \EA(3){4}
            \SO[L=1](1){1'} \SO[L=2](2){2'} \SO[L=3](3){3'} \SO[L=4](4){4'}
            \Edge(1.south)(2'.north) \Edge(2.south)(3'.north) \Edge(3.south)(1'.north) \Edge(4.south)(4'.north)
            \tkzDefPoint(-0.5,-0.5){lab}\tkzLabelPoint[left](lab){$\sigma$}
            \onslide<3->{\SO[L=1](1'){1''} \SO[L=2](2'){2''} \SO[L=3](3'){3''} \SO[L=4](4'){4''}
                \Edge(1'.south)(3''.north) \Edge(3'.south)(1''.north) \Edge(2'.south)(4''.north) \Edge(4'.south)(2''.north)
                \tkzDefPoint(-0.5,-1.5){lab2}\tkzLabelPoint[left](lab2){$\tau$}
            }
            \onslide<4->{
                \EA[unit=3,L=1](4){c1} \EA[L=2](c1){c2} \EA[L=3](c2){c3} \EA[L=4](c3){c4}
                \EA[unit=3,L=1](4''){c1'} \EA[L=2](c1'){c2'} \EA[L=3](c2'){c3'} \EA[L=4](c3'){c4'}
                \onslide<5->{\Edge(c1.south)(c4'.north)}
                \onslide<6->{\Edge(c4.south)(c2'.north)}
                \onslide<7->{\Edge(c2.south)(c1'.north)}
                \onslide<8->{\Edge(c3.south)(c3'.north)}
                \tkzDefPoint(5.5,-1){lab3}\tkzLabelPoint[left](lab3){$\sigma\tau$}
            }
        \end{tikzpicture}
    \end{center}
    \onslide<4->{$$\sigma\tau = (1,2,3)(1,3)(2,4) = (1,\onslide<5->{4,\onslide<6->{2\onslide<7->{)\onslide<8->{\in \Sym(4).}}}}$$}}

    \vspace{-0.5cm}
    \textit{Note:} here, $\sigma\tau \neq \tau\sigma$, since $1^{\sigma\tau} = 4$ but $1^{\tau\sigma} = (1^\tau)^\sigma = 3^\sigma = 1$. Identity $1 = ()$ satisfies $1 \sigma = \sigma 1 = \sigma$ for $\sigma \in \Sym(n)$.
\end{slide}

\subsection{Permutation groups}

\begin{slide}
    \textit{Note:} for $g,h,k \in \Sym(n)$, (i) $gh \in \Sym(n)$, (ii) $1 = () \in \Sym(n)$, (iii) $g^{-1} \in \Sym(n)$, (iv) $(gh)k = g(hk)$. If true for subset:

    \begin{definition}[permutation group]
        \vspace{0pt}
        \textbf{Permutation group} of degree $n$ is subset $G \leq \Sym(n)$ satisfying:
        \begin{enumerate}[(i)]
            \item \textbf{(closure)} $gh \in G$ for $g,h \in G$; \pause
            \item \textbf{(identity)} $1 = () \in G$; \pause
            \item \textbf{(inverses)} $g^{-1} \in G$ for $g \in G$.
        \end{enumerate}
    \end{definition}

    \begin{example}[alternating group]
        \vspace{0pt}
        \textbf{Alternating group} $\Alt(3) = \{(),(1,2,3),(1,3,2)\} < \Sym(3)$. \\
        In general, $\Alt(n)$ is all \textit{even} permutations of $[n]$ (product of even \# of \textit{transpositions} $(i,j)$, e.g. $(1,2,3) = (1,2)(1,3)$).
    \end{example}
\end{slide}

% \subsection{Order of permutations}

% \begin{slide}
%     \begin{definition}[order]
%         \vspace{0pt}
%         \textbf{Order} of $g \in G$ is least $k \in \Z_+$ with $g^k = g \dotsb g = 1$.
%     \end{definition} \pause

%     \begin{example}
%         \vspace{0pt}
%         Consider $g = (1,4,5)(2,6) \in \Sym(6)$.
%         \begin{center}
%             \begin{tikzpicture}[x=0.6cm,y=0.6cm]
%                 \GraphInit[vstyle=Normal]
%                 \tikzset{VertexStyle/.style={draw=none}}
%                 \Vertices{circle}{5,1,4}
%                 \Vertex[x=3,y=0]{2}
%                 \Vertex[x=5,y=0]{6}
%                 \Vertex[x=7,y=0]{3}
%                 \Loop[dist=1cm,dir=EA,style={thick,->}](3)
%                 \tikzset{EdgeStyle/.style={->,bend right}}
%                 \Edges(1,4,5,1)
%                 \Edges[style={bend right=60}](2,6,2)
%                 \tkzDefPoint(-3,0){lab}\tkzLabelPoint[centered](lab){$g$}
%             \end{tikzpicture}
%         \end{center} \pause
%         Then $1^{g^3} = 4^{g^2} = 5^g = 1$, \pause $4^{g^3} = 4$, $5^{g^3} = 5$, $2^{g^2} = 2$, $6^{g^2} = 6$ so \pause $g^6 = () = 1$; order of $g$ is 6. \pause
%     \end{example}

%     \begin{proposition}
%         Order of $g \in \Sym(n)$ is lcm of cycle lengths.
%     \end{proposition}
% \end{slide}

\subsection{Generating a group}

\begin{slide}
    \begin{definition}[generator]
        \vspace{0pt}
        Set $X$ \textbf{generates} $G$ if every $g \in G$ is $g = x_1^{\pm 1} \dotsb x_r^{\pm 1}$ for some $r \in \N$, $x_i \in X$ \textbf{generators}; write $G = \langle X \rangle$. \pause

        (If $G = \langle X \rangle$ for some $X$ with $|X| = 1$, $G$ is \textbf{cyclic}.)
    \end{definition} \pause

    \begin{example}[cyclic group]
        \vspace{0pt}
        Consider $\Alt(3) = \{(),(1,2,3),(1,3,2)\}$: $(1,2,3)^2 = (1,3,2)$, $(1,2,3)^3 = ()$, so $\Alt(3) = \langle(1,2,3)\rangle$ is cyclic (only for $n = 3$).
    \end{example}

    \begin{example}[symmetric group]
        \vspace{0pt}
        Consider $\Sym(3) = \{(),(1,2),(1,3),(2,3),(1,2,3),(1,3,2)\}$. \\
        Not cyclic, but $\Sym(3) = \langle(1,2),(2,3)\rangle$ (adjacent swaps). \\
        Also, $\Sym(3) = \langle(1,2),(1,2,3)\rangle$, e.g. $(2,3) = (1,2,3)(1,2)$.
    \end{example}
\end{slide}

\subsection{Group actions}

\begin{slide}
    \begin{definition}[group action]
        \vspace{0pt}
        For (perm) group $G$ and set $\Omega \neq \emptyset$, a \textbf{$G$-action} is map $\Omega \times G \to \Omega$, $(\alpha,g) \mapsto \alpha^g$ s.t. $\alpha^1 = \alpha$ and $\alpha^{gh} = (\alpha^g)^h$ for $\alpha \in \Omega$ and $g,h \in G$. \textbf{Degree} of action is $|\Omega|$.
    \end{definition}

    \textit{Idea:} $\alpha \in \Omega$ is \textit{state}, apply \textit{move} $g \in G$ to get state $\alpha^g \in \Omega$, in way that respects permutation product. \pause

    \begin{example}[natural action]
        \vspace{0pt}
        $G \leq \Sym(n)$ acts on $\Omega = [n]$ by $\alpha^g := \alpha^g$ (image) for $\alpha \in [n]$, $g \in G$.
    \end{example} \pause

    \begin{example}[right regular action]
        \vspace{0pt}
        Perm group $G$ acts on $\Omega = G$ (itself) via $\alpha^g := \alpha g$ for $\alpha,g \in G$. (\textit{Check:} $\alpha^1 = \alpha 1 = \alpha$ and $\alpha^{gh} = \alpha(gh) = (\alpha g)h = (\alpha^g)^h$.)
    \end{example}
\end{slide}

\begin{slide}
    \begin{example}[dihedral group]
        \vspace{0pt}
        Let $r = (1,2,3,4),s = (1,4)(2,3) \in \Sym(4)$. \textbf{Dihedral group} is $D_8 := \langle r,s \rangle = \{1,r,r^2,r^3,s,sr,sr^2,sr^3\}$, \textit{``symmetries of square''}.

        \onslide<2->{\textit{Note:} $sr = (2,4)$, $sr^2 = (1,2)(3,4)$. Action of $D_8$ on vertices of square (labelled by $[4]$): $g \in D_8$ sends vertex at $i$ to $i^g$.}

        \begin{center}
            \begin{tikzpicture}[x=1cm,y=1cm]
                \GraphInit[vstyle=Classic]
                \tikzset{VertexStyle/.append style={minimum size=1pt}}
                % TOP LEFT
                \tkzDefPoint(0.5,0.25){l2}
                \tkzDefPoint(0.5,-1.25){l3}
                \tkzDefPoint(-0.25,0.25){l4}
                \tkzDefPoint(1.25,-1.25){l5}
                \tikzset{EdgeStyle/.style={dotted}}
                \onslide<5->{\Edge(l2)(l3)} % s
                \onslide<7->{\Edge[color=metrogreen](l4)(l5)} % sr
                \tikzset{EdgeStyle/.style={-}}
                \onslide<3->{{\Vertex[Lpos=180]{1} \SO[Lpos=180](1){2} \EA(2){3} \NO(3){4}
                            \AddVertexColor{red}{1}
                            \AddVertexColor{green}{2}
                            \AddVertexColor{blue}{3}
                            \AddVertexColor{yellow}{4}}
                    \Edges(1,2,3,4,1)
                    \tikzset{EdgeStyle/.style={->}}
                    \tkzDefPoint(1.5,-0.5){a1}
                    \tkzDefPoint(3.5,-0.5){a2}
                    \onslide<4->{\Edge[label=$r$,labelcolor=metropolis](a1)(a2)
                        \tkzDefPoint(2.5,0){l1}\tkzLabelPoint[centered](l1){$\curvearrowleft$}
                        % TOP RIGHT
                        \tkzDefPoint(3.75,-0.5){l2}
                        \tkzDefPoint(5.25,-0.5){l3}
                        \tikzset{EdgeStyle/.style={dotted}}
                        \onslide<8->{\Edge(l2)(l3)} % sr^2
                        \tikzset{EdgeStyle/.style={-}}
                        \onslide<4->{\Vertex[x=4,y=0,Lpos=180]{1} \SO[Lpos=180](1){2} \EA(2){3} \NO(3){4}
                            \AddVertexColor{yellow}{1}
                            \AddVertexColor{red}{2}
                            \AddVertexColor{green}{3}
                            \AddVertexColor{blue}{4}
                            \Edges(1,2,3,4,1)}
                        \tikzset{EdgeStyle/.style={->}}
                        \tkzDefPoint(0.5,-1.5){a1}
                        \tkzDefPoint(0.5,-2.5){a2}
                        \onslide<5->{\Edge[label=$s$,labelcolor=metropolis](a1)(a2)}
                        % BOTTOM LEFT
                        \tikzset{EdgeStyle/.style={-}}
                        \onslide<5->{\Vertex[x=0,y=-3,Lpos=180]{1} \SO[Lpos=180](1){2} \EA(2){3} \NO(3){4}
                            \AddVertexColor{yellow}{1}
                            \AddVertexColor{blue}{2}
                            \AddVertexColor{green}{3}
                            \AddVertexColor{red}{4}
                            \Edges(1,2,3,4,1)}
                        \tikzset{EdgeStyle/.style={->}}
                        \tkzDefPoint(1.5,-3.5){a1}
                        \tkzDefPoint(3.5,-3.5){a2}
                        \onslide<6->{\Edge[label=$r$,labelcolor=metropolis](a1)(a2)
                            \tkzDefPoint(2.5,-3){l1}\tkzLabelPoint[centered](l1){$\curvearrowleft$}}
                        % BOTTOM RIGHT
                        \tikzset{EdgeStyle/.style={-}}
                        \onslide<6->{\Vertex[x=4,y=-3,Lpos=180]{1} \SO[Lpos=180](1){2} \EA(2){3} \NO(3){4}
                            \AddVertexColor{red}{1}
                            \AddVertexColor{yellow}{2}
                            \AddVertexColor{blue}{3}
                            \AddVertexColor{green}{4}
                            \Edges(1,2,3,4,1)}
                        \tikzset{EdgeStyle/.style={->}}
                        \tkzDefPoint(4.5,-1.5){a1}
                        \tkzDefPoint(4.5,-2.5){a2}
                        \onslide<8->{\Edge[label=$sr^2$,labelcolor=metropolis](a1)(a2)}
                        \tkzDefPoint(1.5,-1.5){a1}
                        \tkzDefPoint(3.5,-2.5){a2}
                        \onslide<7->{\Edge[label=$sr$,labelcolor=metropolis,labeltext=metrogreen,color=metrogreen](a1)(a2)}}}
            \end{tikzpicture}
        \end{center}
    \end{example}
\end{slide}

\subsection{Orbits and stabilisers}

\begin{slide}
    \begin{definition}[orbit]
        \vspace{0pt}
        If $G$ acts on $\Omega$, then \textbf{orbit} of $\alpha \in \Omega$ is $\alpha^G := \{\alpha^g : g \in G\}$. \\
        \textit{Idea:} states $\alpha^g \in \Omega$ reachable from fixed $\alpha \in \Omega$ by moves $g \in G$. \pause
    \end{definition}

    \begin{definition}[stabiliser]
        \vspace{0pt}
        If $G$ acts on $\Omega$, then \textbf{stabiliser} of $\alpha \in \Omega$ is $G_\alpha := \{g \in G : \alpha^g = \alpha\}$. \\
        \textit{Idea:} moves $g \in G$ that fix given $\alpha \in \Omega$. \pause
    \end{definition}

    \begin{example}[natural action]
        \vspace{0pt}
        $G = \Alt(3) = \{(),(1,2,3),(1,3,2)\}$ acts on $\Omega = [3]$ naturally. \\
        Orbit of $1$ is \pause $1^G = \{1,2,3\} = [3]$; stabiliser of $1$ is \pause $G_1 = \{()\} = 1$.

        One orbit only: \textbf{transitive} action.
    \end{example}
\end{slide}

\begin{slide}
    Orbit $\alpha^G$: states $\alpha^g \in \Omega$ reachable from fixed $\alpha$ by moves $g \in G$. \\
    Stabiliser $G_\alpha$: moves $g \in G$ that fix given $\alpha$.

    \begin{example}[dihedral group]
        \vspace{0pt}
        Recall $G = D_8 = \langle r,s \rangle = \{1,r,r^2,r^3,s,sr,sr^2,sr^3\} \leq \Sym(4)$ where $r = (1,2,3,4)$, $s = (1,4)(2,3)$.

        Orbit of 1: $1^1 = 1$, $1^r = 2$, $1^{r^2} = 3$, $1^{r^3} = 4$, so $1^G = [4]$ (transitive).

        Stabiliser of 1: $sr = (2,4)$, $sr^2 = (1,2)(3,4)$, $sr^3 = (1,3)$, so $G_1 = \{(),(2,4)\} = \{1,sr\}$.

        \textit{Note:} $|1^G||G_1| = 4 \cdot 2 = 8 = |G|$. Coincidence?
    \end{example}

    \begin{theorem}[orbit-stabiliser]
        \vspace{0pt}
        If $G$ acts on $\Omega$, then for $\alpha \in \Omega$, $|\alpha^G||G_\alpha| = |G|$.
    \end{theorem}
\end{slide}

\subsection{Blocks and primitivity}

\begin{slide}
    \begin{definition}[block]
        \vspace{0pt}
        If $G$ acts transitively on $\Omega$ and $\Delta \subseteq \Omega$, let $\Delta^g := \{\alpha^g : \alpha \in \Delta\}$. \\
        A \textbf{block} is $\Delta \subseteq \Omega$ with $\Delta^g = \Delta$ or $\Delta^g \cap \Delta = \emptyset$ for all $g \in G$.

        Block is \textbf{nontrivial} if $|\Delta| > 1$ and $\Delta \neq \Omega$.
    \end{definition}

    \textit{Examples of blocks:} singletons, $\Omega$, orbits.

    \begin{definition}[primitivity]
        \vspace{0pt}
        A \textit{transitive} $G$-action is \textbf{primitive} if there are no nontrivial blocks; otherwise it is \textbf{imprimitive}.

        If $G$ is perm group with primitive natural action, $G$ is \textbf{primitive}.
    \end{definition}

    For block $\Delta$, define \textbf{block system} $\Sigma = \{\Delta^g : g \in G\}$ (partitions $\Omega$); then $G$ acts on $\Sigma$; if $\Delta$ is \textit{maximal}, then acts primitively.
\end{slide}

\begin{slide}
    \begin{example}[dihedral group]
        \vspace{0pt}
        Recall $G = D_8 = \langle r,s \rangle = \{1,r,r^2,r^3,s,sr,sr^2,sr^3\} \leq \Sym(4)$ where $r = (1,2,3,4)$, $s = (1,4)(2,3)$, $sr = (2,4)$.

        Block is $\Delta = \{1,3\}$ (nontrivial) with block system $\Sigma = \{\{1,3\},\{2,4\}\}$ (opposite vertices stay opposite):
        \begin{center}
            \begin{tikzpicture}[x=1cm,y=1cm]
                \GraphInit[vstyle=Classic]
                \tikzset{VertexStyle/.append style={minimum size=1pt}}
                % TOP LEFT
                \tkzDefPoint(0.5,0.25){l2}
                \tkzDefPoint(0.5,-1.25){l3}
                \tkzDefPoint(-0.25,0.25){l4}
                \tkzDefPoint(1.25,-1.25){l5}
                \tikzset{EdgeStyle/.style={dotted}}
                \tikzset{EdgeStyle/.style={-}}
                {\Vertex[Lpos=180]{1} \SO[Lpos=180](1){2} \EA(2){3} \NO(3){4}
                    \AddVertexColor{red}{1}
                    \AddVertexColor{green}{2}
                    \AddVertexColor{red}{3}
                    \AddVertexColor{green}{4}}
                \Edges(1,2,3,4,1)
                \tikzset{EdgeStyle/.style={->}}
                \tkzDefPoint(1.5,-0.5){a1}
                \tkzDefPoint(3.5,-0.5){a2}
                \Edge[label=$r$,labelcolor=metropolis](a1)(a2)
                \tkzDefPoint(2.5,0){l1}\tkzLabelPoint[centered](l1){$\curvearrowleft$}
                % TOP RIGHT
                \tkzDefPoint(3.75,-0.5){l2}
                \tkzDefPoint(5.25,-0.5){l3}
                \tikzset{EdgeStyle/.style={dotted}}
                \tikzset{EdgeStyle/.style={-}}
                \Vertex[x=4,y=0,Lpos=180]{1} \SO[Lpos=180](1){2} \EA(2){3} \NO(3){4}
                \AddVertexColor{green}{1}
                \AddVertexColor{red}{2}
                \AddVertexColor{green}{3}
                \AddVertexColor{red}{4}
                \Edges(1,2,3,4,1)
            \end{tikzpicture}
        \end{center}
        e.g. $\Delta^r = \{2,4\}$, $\Delta^s = \{4,2\}$, $\Delta^{sr} = \{1,3\} = \Delta$.

        $D_8$ acts imprimitively on $[4]$ but primitively on $\Sigma$ (degree 2).
    \end{example}
\end{slide}

\section{The Rubik's group}

\subsection{Representing the cube and its operations}

\begin{slide}
    Rubik's cube has 6 faces, each with $3 \times 3$ small \textit{stickers}.

    \onslide<2->{In \textbf{solved state} $1$, label stickers (except each centre) using $[48]$:}

    \begin{center}
        \includegraphics<1|handout:0>{graphics/rubiks_cube_net_empty.tikz}%
        \includegraphics<2->{graphics/rubiks_cube_net.tikz}%
    \end{center}

    \onslide<3->{6 \textbf{generators} (\textit{moves} in CC): $U,L,F,R,B,D$ (rot. \textit{clockwise}).}
\end{slide}

\begin{slide}
    From \textit{solved state} $1$, consider $F$ which rotates front face clockwise:

    \begin{center}
        \includegraphics<1|handout:0>{graphics/rubiks_cube_net.tikz}%
        \includegraphics<2->{graphics/rubiks_cube_net_front.tikz}%
    \end{center}

    \vspace{-1cm}
    \onslide<2->{\begin{multline*}
            F = (17,19,24,22)(18,21,23,20)( 6,25,43,16)\\
            ( 7,28,42,13)( 8,30,41,11) \in \Sym(48).
        \end{multline*}}
\end{slide}

\subsection{The Rubik's group of permutations}

\begin{slide}
    Generators as permutations of labels $[48]$:

    {\scriptsize
    \begin{itemize}
        \item $U = ( 1, 3, 8, 6)( 2, 5, 7, 4)( 9,33,25,17)(10,34,26,18)(11,35,27,19)$
        \item $L = ( 9,11,16,14)(10,13,15,12)( 1,17,41,40)( 4,20,44,37)( 6,22,46,35)$
        \item $F = (17,19,24,22)(18,21,23,20)( 6,25,43,16)( 7,28,42,13)( 8,30,41,11)$
        \item $R = (25,27,32,30)(26,29,31,28)( 3,38,43,19)( 5,36,45,21)( 8,33,48,24)$
        \item $B = (33,35,40,38)(34,37,39,36)( 3, 9,46,32)( 2,12,47,29)( 1,14,48,27)$
        \item $D = (41,43,48,46)(42,45,47,44)(14,22,30,38)(15,23,31,39)(16,24,32,40)$
    \end{itemize}} \pause

    \textbf{Operation} is sequence of generators and inverses. E.g. $RUR^{-1}U^{-1}$, \pause $URU^{-1}L^{-1}UR^{-1}U^{-1}L$, \pause $RUR^{-1}URU^2R^{-1}U^2$, \pause $1 = ()$.

    \begin{definition}[Rubik's group]
        \vspace{0pt}
        $\RC = \langle U,L,F,R,B,D \rangle \leq \Sym(48)$ is permutation group of degree 48, called \textbf{Rubik's group}.
    \end{definition}

    Clearly $\RC$ is finite, but what is $|\RC|$?
\end{slide}

% \begin{slide}
%     \textit{In cubing community:} operations called \textit{move sequences}. Inverse generators (also \textit{``moves''} in CC) written $U',L',F',R',B',D'$ (instead of $U^{-1}$ etc.); powers written $U2,R2$ etc. (instead of $U^2,R^2$). \pause

%     \textit{Recall:} $\sigma = \tau$ in $\Sym(n)$ iff $i^\sigma = i^\tau$ for all $i \in [n]$. \pause

%     Operations \textit{don't generally commute}: $RU \neq UR$ since

%         {\scriptsize
%             \begin{itemize}
%                 \item $R = (25,27,32,30)(26,29,31,28)( 3,38,43,19)( 5,36,45,21)( 8,33,48,24)$
%                 \item $U = ( 1, 3, 8, 6)( 2, 5, 7, 4)( 9,33,25,17)(10,34,26,18)(11,35,27,19)$
%             \end{itemize}}

%     \vspace{-1cm}
%     $$19^{RU} = \pause (19^R)^U = \pause 3^U = \pause 8 \quad\text{but}\quad 19^{UR} = \pause (19^U)^R = \pause 11^R = \pause 11.$$
% \end{slide}

\begin{slide}
    \texttt{GAP} code to define generators and $\RC = \langle U,L,F,R,B,D \rangle$ (as \texttt{G}):

    {\footnotesize\lstinputlisting{code/rubiks_def.gap}} \pause

    \texttt{Order} cmd: $|\RC| = 43\,252\,003\,274\,489\,856\,000 \approx 4.3 \cdot 10^{19}$. \textit{How?}
\end{slide}

\subsection{Orbits and stabilisers}

% UPDATE TOGETHER WITH BELOW
\begin{slide}
    \begin{overprint}
        \begin{center}
            \scalebox{0.6}{\includegraphics{graphics/rubiks_cube_net.tikz}}
        \end{center}

        \onslide<1>
        \scriptsize\lstinputlisting{code/rubiks_orbit_stab.gap}

        \normalsize Two $\RC$-orbits: corner pieces $1^\RC$, edge pieces $2^\RC$.

        \onslide<2-|handout:0>
        Moves in $\mathcal{H} = \RC_{1,3,6,8} = (((\RC_1)_3)_6)_8$ fix white corners $1,3,6,8$.

        \onslide<3-|handout:0>{{\tiny\lstinputlisting{code/rubiks_orbit_stab_2.gap}}

                {\footnotesize Some $\mathcal{H}$-orbits: $17^{\mathcal{H}} = \{17\}$, bottom corner stickers $24^{\mathcal{H}}$, edge stickers $2^{\mathcal{H}} = 2^\RC$.}}
    \end{overprint}
\end{slide}

% % UPDATE TOGETHER WITH ABOVE
% \begin{slide}<beamer:0>
%     \begin{overprint}
%         \begin{center}
%             \scalebox{0.6}{\includegraphics{graphics/rubiks_cube_net.tikz}}
%         \end{center}

%         \onslide<1|handout:0>
%         \scriptsize\lstinputlisting{code/rubiks_orbit_stab.gap}

%         \normalsize Two $\RC$-orbits: corner pieces $1^\RC$, edge pieces $2^\RC$.

%         \onslide<2->
%         Moves in $\mathcal{H} = \RC_{1,3,6,8} = (((\RC_1)_3)_6)_8$ fix white corners $1,3,6,8$.

%         \onslide<3->{{\tiny\lstinputlisting{code/rubiks_orbit_stab_2.gap}}

%                 {\footnotesize Some $\mathcal{H}$-orbits: $17^{\mathcal{H}} = \{17\}$, bottom corner stickers $24^{\mathcal{H}}$, edge stickers $2^{\mathcal{H}} = 2^\RC$.}}
%     \end{overprint}
% \end{slide}

\subsection{Transitive action of Rubik's group on corners}

\begin{slide}
    $\RC$ acts transitively on corner stickers $1^\RC$. In this action:

    $FRU$ corner $\{8,25,19\}$ is block (3 stickers never separate);
    \begin{multline*}
        \Sigma = \{\{1,35,9\},\{6,11,17\},\{40,46,14\},\{27,3,33\},\\
        \{8,25,19\},\{16,41,22\},\{32,48,38\},\{24,43,30\}\}
    \end{multline*}
    is system of imprimitivity (containing 8 corners).

    $\RC$ acts primitively on $\Sigma$; result
\end{slide}

\section{Bases and stabiliser chains}

\subsection{Bases and stabiliser chains}

\begin{slide}
    \begin{definition}[Base, stabiliser chain]
        \vspace{0pt}
        If $G \leq \Sym(n)$, distinct elts $B = [\beta_1,\dotsc,\beta_r] \subseteq [n]$ is \textbf{base} for $G$ if $G_{\beta_1,\dotsc,\beta_r} = 1$. (\textit{Recall:} $G_{\beta_1,\dotsc,\beta_r} = \{g \in G : \beta_1^g = \beta_1,\dotsc,\beta_r^g = \beta_r\}$.) \pause

        Corresponding \textbf{stabiliser chain} is
        \[G = G^0 \geq G^1 \geq \dotsb \geq G^r = 1\]
        where $G^i = G^{i-1}_{\beta_i} = G_{\beta_1,\dotsc,\beta_i}$.
    \end{definition} \pause

    Base $B$ contains elts of $[n]$ such that only $1 \in G$ fixes every $\beta_i \in B$. (Short base desirable: how to compute \textbf{min base} of length $b(G)$?) \pause

    \begin{theorem}[Blaha, 1992]
        \vspace{0pt}
        Problem of finding minimum base for $G$ is NP-complete, even for cyclic groups (if $\mathrm{P} \neq \mathrm{NP}$, then no polynomial time algorithm).
    \end{theorem}
\end{slide}

\begin{slide}
    \begin{example}[Rubik's group]
        \vspace{0pt}
        Using \texttt{BaseOfGroup} cmd in \texttt{GAP}, base of $\mathcal{G}$ of size 18 is
        % \lstinputlisting{code/rubiks_group_base.gap} \pause
        $$B = [ 1, 3, 6, 8, 2, 4, 5, 7, 12, 13, 14, 15, 16, 21, 23, 24, 29, 31 ].$$ \pause
        Contains: 7 corner stickers (from 7 of 8 corners), 11 edge stickers (from 11 of 12 edges).
    \end{example}

    \begin{theorem}
        \vspace{0pt}
        For Rubik's group $\RC$, $b(\RC) = 18$.
    \end{theorem}
\end{slide}

\begin{slide}
    Stabiliser chain can be implemented computationally; useful in algorithms.

    Let $G = \langle X \rangle \leq \Sym(n)$ have base $B$ and stabiliser chain
    $$G = G^0 \geq G^1 \geq \dotsb \geq G^r = 1.$$

    \begin{alertblock}{Problem (random element generation)}
        \vspace{0pt}
        Generate uniformly random element of $G$.

        (\textit{Alternative:} random product of generators in $X$; distribution?)
    \end{alertblock}

    \begin{alertblock}{Problem (membership testing)}
        \vspace{0pt}
        For $\sigma \in \Sym(n)$, test if $\sigma \in G$.

        (\textit{Application}: check if restickering of Rubik's cube is valid state.)
    \end{alertblock}
\end{slide}

\subsection{Greedy bases}

\begin{slide}
    % \begin{example}[Rubik's group]
    %     \vspace{0pt}
    %     Using \texttt{GAP}:

    %     \lstinputlisting{code/rubiks_group_base.gap} \pause

    %     Base of $\mathcal{G}$ of size 18 is
    %     $$B = [ 1, 3, 6, 8, 2, 4, 5, 7, 12, 13, 14, 15, 16, 21, 23, 24, 29, 31 ].$$ \pause
    %     If move $\sigma \in \mathcal{G}$ fixes every $\beta_i \in B$ then $\sigma = 1$ is empty move.
    % \end{example}


\end{slide}

\subsection{What is the size of the Rubik's group?}

\begin{slide}
    \begin{theorem}[size of perm group]
        \vspace{0pt}
        If $B = [\beta_1,\dotsc,\beta_r]$ is base for $G \leq \Sym(n)$ with stabiliser chain $G = G^0 \geq G^1 \geq \dotsb \geq G^r = 1$, then
        $$|G| = |\beta_1^{G^0}||\beta_2^{G^1}| \dotsb |\beta_r^{G^{r-1}}|.$$
    \end{theorem}

    \onslide<2->
    Orbits and stabilisers can be easily computed (e.g. using \texttt{GAP}).

    \onslide<3->{Implementing base and stabiliser chain for Rubik's group $\RC$ (using \texttt{BaseOfGroup} and \texttt{StabChain} cmds), \texttt{GAP} computes:}

    \onslide<4->{
        \begin{corollary}
            \vspace{0pt}
            $|\RC| = 43\,252\,003\,274\,489\,856\,000 \approx 4.3 \cdot 10^{19}$.
        \end{corollary}

        % (\textit{Note:} $|\RC| = 2^{27} \cdot 3^{14} \cdot 5^3 \cdot 7^2 \cdot 11$. Thus no move of order 13.)
    }
\end{slide}

\section{Base sizes of primitive groups}

\subsection{Affine groups}

\begin{slide}
    Definition
\end{slide}

\subsection{Large base permutation groups}

\begin{slide}
    Definition
\end{slide}

\begin{slide}
    Liebeck

    Moscatiello, Roney-Dougal
\end{slide}

\subsection{Main result}

\begin{slide}
    Statement
\end{slide}

\begin{slide}
    Approach (dot points/observations)
\end{slide}

\begin{slide}
    Conjecture
\end{slide}

\section{Concluding remarks}

\subsection{References and resources}

\begin{slide}
    \small
    \begin{itemize}
        \item Analyzing Rubik's cube with \texttt{GAP}: \url{https://www.gap-system.org/Doc/Examples/rubik.html}
        \item J. A. Paulos --- \textit{Innumeracy} (book)
        \item Holt --- \textit{Handbook of Computational Group Theory} (textbook)
        \item Dixon and Mortimer --- \textit{Permutation Groups} (textbook)
        \item Blaha --- \textit{Minimum bases for permutation groups: The greedy approximation}, 1992: \url{https://doi:10.1016/0196-6774(92)90020-D}
        \item Liebeck --- \textit{On minimal degrees and base sizes of primitive permutation groups}, 1984: \url{https://doi.org/10.1007/bf01193603}
        \item Moscatiello and Roney-Dougal: \textit{Base sizes of primitive permutation groups}, 2021: \url{https://doi.org/10.1007/s00605-021-01599-5}
    \end{itemize}
\end{slide}

\begin{slide}
    The \textbf{order} of $g \in G \leq \Sym(\Omega)$ is smallest $k \in \Z_+$ such that $g^k = 1$. \textit{Fact:} order of $g$ is lcm of cycle lengths; it divides $|G|$.

    \textit{Note:} for Rubik's group, $R$ has order 4, $RUR^{-1}U^{-1}$ has order 6, $RU$ has order 105 (\texttt{GAP}). Order 7? \pause $(RU)^{15}$. Order 13? \pause None;
    $$|\RC| = 2^{27} \cdot 3^{14} \cdot 5^3 \cdot 7^2 \cdot 11.$$

    \small
    \begin{itemize}
        \item \textit{Bonus:} Orders of elements in Rubik's group (1260 largest, 13 smallest without, 11 rarest, 60 most common, median 67.3, 73 options): \url{https://www.jaapsch.net/puzzles/cubic3.htm\#p34}
        \item \textit{Bonus:} Thistlethwaite's 52 move algorithm (using group theory): \url{https://www.jaapsch.net/puzzles/thistle.htm}
    \end{itemize}
\end{slide}

\subsection{Large base definition}

\begin{slide}
    \begin{definition}
        \vspace{0pt}
        Perm group $G$ of degree $n$ is \textbf{large base} if
        $$\Alt(m)^r \unlhd G \leq \Sym(m) \wr \Sym(r)$$
        for some $m,r,k$, where $\Sym(m)$ acts on $\binom{[m]}{k}$, and if $r > 1$ then wreath product has \textit{product action} of degree $n = \binom{m}{k}^r$.
    \end{definition}
\end{slide}

% \subsection{References}

% \begin{slide}
%     \nocite{*}
%     \renewcommand{\bibname}{References}
%     \bibliographystyle{plain} % BibTeX only
%     \bibliography{references} % BibTeX only
% \end{slide}

\end{document}