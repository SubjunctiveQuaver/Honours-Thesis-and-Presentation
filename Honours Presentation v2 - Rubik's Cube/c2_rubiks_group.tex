\subsection{Representing the cube and its moves}

\begin{slide}
    Rubik's cube has 6 faces, each with $3 \times 3$ small \textit{facelets}.

    \onslide<2->{In \textbf{solved state} $1$, label facelets (except each centre) using $[48]$:}

    \begin{center}
        \includegraphics<1|handout:0>{graphics/rubiks_cube_net_empty.tikz}%
        \includegraphics<2->{graphics/rubiks_cube_net.tikz}%
    \end{center}

    \onslide<3->{6 \textbf{generators} (\textit{moves} in CC): $U,L,F,R,B,D$ (rot. \textit{clockwise}).}
\end{slide}

\begin{slide}
    From \textit{solved state} $1$, consider $F$ which rotates front face clockwise:

    \begin{center}
        \includegraphics<1|handout:0>{graphics/rubiks_cube_net.tikz}%
        \includegraphics<2->{graphics/rubiks_cube_net_front.tikz}%
    \end{center}

    \onslide<2->{\scriptsize{Under $F$: $17 \mapsto 19 \mapsto 24 \mapsto 22 \mapsto 17$, $18 \mapsto 21 \mapsto 23 \mapsto 20 \mapsto 18$, $6 \mapsto 25 \mapsto 43 \mapsto 16 \mapsto 6$, $7 \mapsto 28 \mapsto 42 \mapsto 13 \mapsto 7$, $8 \mapsto 30 \mapsto 41 \mapsto 11 \mapsto 8$, else fixed. So
            \onslide<3->{$$F = (17,19,24,22)(18,21,23,20)( 6,25,43,16)( 7,28,42,13)( 8,30,41,11) \in \Sym(48).$$}}}
\end{slide}

\begin{slide}
    Generators as permutations of labels $[48]$:

    {\scriptsize
    \begin{itemize}
        \item $U = ( 1, 3, 8, 6)( 2, 5, 7, 4)( 9,33,25,17)(10,34,26,18)(11,35,27,19)$
        \item $L = ( 9,11,16,14)(10,13,15,12)( 1,17,41,40)( 4,20,44,37)( 6,22,46,35)$
        \item $F = (17,19,24,22)(18,21,23,20)( 6,25,43,16)( 7,28,42,13)( 8,30,41,11)$
        \item $R = (25,27,32,30)(26,29,31,28)( 3,38,43,19)( 5,36,45,21)( 8,33,48,24)$
        \item $B = (33,35,40,38)(34,37,39,36)( 3, 9,46,32)( 2,12,47,29)( 1,14,48,27)$
        \item $D = (41,43,48,46)(42,45,47,44)(14,22,30,38)(15,23,31,39)(16,24,32,40)$
    \end{itemize}} \pause

    \textbf{(Valid) move} is sequence of generators and inverses. E.g. $RUR^{-1}U^{-1}$, \pause $URU^{-1}L^{-1}UR^{-1}U^{-1}L$, \pause $RUR^{-1}URU^2R^{-1}U^2$.

    \textbf{Empty move} is $1 = ()$ (valid: $1 = RR^{-1}$). \pause

    \textbf{Solving} is applying valid move to get to solved state $1$.
\end{slide}

\begin{slide}
    \textit{In cubing community:} moves called \textit{move sequences}. Inverse generators (also \textit{``moves''} in CC) written $U',L',F',R',B',D'$ (instead of $U^{-1}$ etc.); powers written $U2,R2$ etc. (instead of $U^2,R^2$). \pause

    \textit{Recall:} $\sigma = \tau$ in $\Sym(n)$ iff $i^\sigma = i^\tau$ for all $i \in [n]$. \pause

    Moves \textit{don't generally commute}: $RU \neq UR$ since

        {\scriptsize
            \begin{itemize}
                \item $R = (25,27,32,30)(26,29,31,28)( 3,38,43,19)( 5,36,45,21)( 8,33,48,24)$
                \item $U = ( 1, 3, 8, 6)( 2, 5, 7, 4)( 9,33,25,17)(10,34,26,18)(11,35,27,19)$
            \end{itemize}}

    \vspace{-1cm}
    $$19^{RU} = \pause (19^R)^U = \pause 3^U = \pause 8 \quad\text{but}\quad 19^{UR} = \pause (19^U)^R = \pause 11^R = \pause 11.$$
\end{slide}

\subsection{Moves vs states for Rubik's cube}

\begin{slide}
    \textbf{(Valid) state} is result of applying \textit{valid move} to \textit{solved state} $1$.

    \begin{center}
        \includegraphics<1|handout:0>{graphics/rubiks_cube_net.tikz}%
        \includegraphics<2->{graphics/rubiks_cube_net_front.tikz}%
    \end{center}

    \onslide<2->{This new state is valid, as result of applying $F$ to solved state.}
\end{slide}

\begin{slide}
    \textit{Restickering} is valid state iff it can be \textit{solved}. How to check?

    Let $\RS$ be valid \textbf{states}; let state $x \in \RS$ be element of $\Sym(48)$ giving permutation of labels to solved state $1 \in \RS$. \\ (I.e. $i^x$ is $1$-label of facelet at $x$-position of facelet $i$.) \pause

    Let $\RC$ be valid \textbf{moves}; each generator/inverse applied corresponds to turn of that face (determined by centre facelet), holding cube fixed. \pause

    \begin{itemize}
        \item State $x \in \RS$ corresponds to move $x \in \RC$ required to get solved state $1$ into state $x$. \pause
        \item Move $\sigma \in \RC$ corresponds to state $\sigma \in \RS$ reached by applying move $\sigma$ to solved state $1$.
    \end{itemize} \pause

    So moves $\leftrightarrow$ states; as sets, $\RS = \RC$. \textit{Solved state} is $1 = () \in \Sym(48)$.
\end{slide}

\subsection{The Rubik's group of permutations}

\begin{slide}
    For set of moves $\RC$: product of valid moves is valid move; identity $1 = () \in \RC$, inverse moves exist (undo generators/inverses). \pause

    \begin{definition}[Rubik's group]
        \vspace{0pt}
        $\RC \leq \Sym(48)$ is permutation group of degree 48, called \textbf{Rubik's group}. \textit{Note:} $G = \langle U,L,F,R,B,D \rangle$.
    \end{definition} \pause

    $\RC$ acts on non-centre facelets labelled by $[48]$: for $\sigma \in \RC$, $i^\sigma$ is $1$-label on facelet that $\sigma$ sends facelet $i$ to, from solved state $1$. (This corresponds to \textit{natural action} as perm group; c.f. $D_8$-action earlier.) \pause

    For \textit{move} $\sigma \in \RC$ and \textit{state} $x \in \RS$, applying $\sigma$ to $x$ gives \textit{state} $x^\sigma = x\sigma \in \RS$. This is \textit{regular action} of $\RC$. (Consider states $x \in \RC$.) \pause

    Clearly $\RC$ finite (states $\leftrightarrow$ moves; also $|\RC| \leq 48!$). But what is $|\RC|$?
\end{slide}

\begin{slide}
    \texttt{GAP} code to define generators and $\RC = \langle U,L,F,R,B,D \rangle$ (as \texttt{G}):

    {\footnotesize\lstinputlisting{code/rubiks_def.gap}} \pause

    \texttt{Order} cmd: $|\RC| = 43\,252\,003\,274\,489\,856\,000 \approx 4.3 \cdot 10^{19}$. \textit{How?}
\end{slide}

% \subsection{Orbits and stabilisers}

% % UPDATE TOGETHER WITH BELOW
% \begin{slide}
%     \begin{overprint}
%         \begin{center}
%             \scalebox{0.6}{\includegraphics{graphics/rubiks_cube_net.tikz}}
%         \end{center}

%         \onslide<1>
%         \scriptsize\lstinputlisting{code/rubiks_orbit_stab.gap}

%         \normalsize Two $\RC$-orbits: corner pieces $1^\RC$, edge pieces $2^\RC$.

%         \onslide<2-|handout:0>
%         Moves in $\mathcal{H} = \RC_{1,3,6,8} = (((\RC_1)_3)_6)_8$ fix white corners $1,3,6,8$.

%         \onslide<3-|handout:0>{{\tiny\lstinputlisting{code/rubiks_orbit_stab_2.gap}}

%                 {\footnotesize Some $\mathcal{H}$-orbits: $17^{\mathcal{H}} = \{17\}$, bottom corner pieces $24^{\mathcal{H}}$, edge pieces $2^{\mathcal{H}} = 2^\RC$.}}
%     \end{overprint}
% \end{slide}

% % UPDATE TOGETHER WITH ABOVE
% \begin{slide}<beamer:0>
%     \begin{overprint}
%         \begin{center}
%             \scalebox{0.6}{\includegraphics{graphics/rubiks_cube_net.tikz}}
%         \end{center}

%         \onslide<1|handout:0>
%         \scriptsize\lstinputlisting{code/rubiks_orbit_stab.gap}

%         \normalsize Two $\RC$-orbits: corner pieces $1^\RC$, edge pieces $2^\RC$.

%         \onslide<2->
%         Moves in $\mathcal{H} = \RC_{1,3,6,8} = (((\RC_1)_3)_6)_8$ fix white corners $1,3,6,8$.

%         \onslide<3->{{\tiny\lstinputlisting{code/rubiks_orbit_stab_2.gap}}

%                 {\footnotesize Some $\mathcal{H}$-orbits: $17^{\mathcal{H}} = \{17\}$, bottom corner pieces $24^{\mathcal{H}}$, edge pieces $2^{\mathcal{H}} = 2^\RC$.}}
%     \end{overprint}
% \end{slide}

% \subsection{Orders of moves}

% \begin{slide}
%     Use \texttt{GAP} to compute products, order (using \texttt{Order} cmd).

%         {\scriptsize\lstinputlisting{code/rubik_product.gap}}

%     How many times must we repeat move $\sigma \in \RC$ to have no effect? \pause I.e. for state $x \in \RS$, smallest $k \in \Z_+$ with $x^{\sigma^k} = x\sigma^k = x \iff \sigma^k = 1$. \pause \textit{Recall:} order of $\sigma$ is lcm of cycle lengths.

%     \begin{itemize}
%         \item Any \textit{generator} ($U,L,F,R,B,D$) has cycles of length $4,4,4,4,4$: order is $\lcm(4,4,4,4,4) = 4$. \pause
%         \item \textit{Commutator} $RUR^{-1}U^{-1}$
%               $$= (1,27,35,33,9,3)(2,21,5)(8,30,25,43,19,24)(26,34,28){:}$$
%               order is $\lcm(6,3,6,3) = 6$.
%     \end{itemize}
% \end{slide}

% \begin{slide}
%     \begin{itemize}
%         \item \textit{Sune} $RUR^{-1}URU^2R^{-1}U^2$
%               $$= (1,9,35)(2,5,7)(3,33,27)(8,25,19)(18,34,26){:}$$
%               order is $\lcm(3,3,3,3,3) = 3$. \pause
%         \item \textit{Lawrence's move} $RU$ has cycles of length $15,7,3,7$: order is $\lcm(15,7,3,7) = 105$. \pause
%         \item \textit{Clayton's move} $UL'$ has cycles of length $9,7,9,7$: order is $\lcm(9,7,9,7) = 63$.
%     \end{itemize}

%     Watch video demonstration by my friend Wes :D \pause

%     Move of order 5? \pause \textit{Answer:} $(RU)^{21}$ since $((RU)^{21})^5 = (RU)^{105} = 1$.

%     What is \textit{smallest} $k \in \Z_+$ with no move of that order?
% \end{slide}

% \subsection{Jake's theorems}

% \begin{slide}
%     \begin{theorem}[Jake Vandenberg's conjecture]
%         \vspace{0pt}
%         There is no Rubik's cube move that cycles through all states.
%     \end{theorem} \pause

%     \textit{Recall:} states $\leftrightarrow$ moves. Rubik's group $\RC$ acts on states by applying move $\sigma \in \RC$ to state $x \in \RC$ to get state $x^\sigma = x\sigma \in \RC$. \pause

%     \textit{Equivalent question:} for starting state, WLOG $1 = ()$, is there $\sigma \in \RC$ with $\{1^{\sigma^k} : k \in \Z\} = \{1\sigma^k : k \in \Z\} = \{\sigma^k : k \in \Z\} = \RC$? \pause In group theory language:

%     \begin{theorem}[Jake Vandenberg's conjecture]
%         \vspace{0pt}
%         The Rubik's group $\RC$ is not cyclic. (I.e. no $\sigma \in \RC$ with $\RC = \langle \sigma \rangle$.)
%     \end{theorem} \pause

%     \begin{proof}
%         \vspace{0pt}
%         If $\RC$ is cyclic, then $\RC$ is abelian. \pause But $\RC$ is not abelian: $RU \neq UR$.
%     \end{proof}
% \end{slide}

% % \begin{slide}
% %     \begin{theorem}[Jake Vandenberg's theorem]
% %         \vspace{0pt}
% %         There is no Rubik's cube move that when repeated, if starting from the solved state, never returns to the solved state.
% %     \end{theorem} \pause

% %     $k$-fold repetition of move $\sigma \in G$, applied to solved state $1 = ()$, gives $1^{\sigma^k} = 1\sigma^k = \sigma^k$. Returning to solved state: $\sigma^k = 1$ (for $k \in \Z_+$). \pause

% %     \textit{Equivalent question:} does any $\sigma \in G$ have infinite order? \pause

% %     \begin{proposition}[corollary of Lagrange's theorem]
% %         \vspace{0pt}
% %         If $G$ is finite group and $g \in G$, then order of $g$ divides $|G|$. So $g^{|G|} = 1$.
% %     \end{proposition} \pause

% %     \begin{corollary}[Jake Vandenberg's theorem]
% %         \vspace{0pt}
% %         There is no $\sigma \in \RC$ with infinite order (since $\RC$ is finite).
% %     \end{corollary}
% % \end{slide}