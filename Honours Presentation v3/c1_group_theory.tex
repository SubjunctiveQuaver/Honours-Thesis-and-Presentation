\subsection{Permutations}

\begin{slide}
    \begin{definition}[permutation]
        \vspace{0pt}
        \textbf{Permutation} of $[n] := \{1,\dotsc,n\}$ is bijection $\sigma : [n] \to [n]$. \textbf{Symmetric group} $\Sym(n)$ is set of permutations of $[n]$.
    \end{definition}

    \onslide<2->{Write $1 = ()$ for identity. Write $i^\sigma$ not $\sigma(i)$ for \textit{image}.

        \onslide<3->{\textit{Cycle notation:} $\sigma = (1,4,5)(2,6) \in \Sym(6)$ is:

            \begin{center}
                \begin{tikzpicture}[x=1cm,y=2cm]
                    \GraphInit[vstyle=Normal]
                    \tikzset{VertexStyle/.style={draw=none}}
                    \tikzset{EdgeStyle/.style={->}}
                    \Vertex{1} \EA(1){2} \EA(2){3} \EA(3){4} \EA(4){5} \EA(5){6}
                    \SO[L=1](1){1'} \SO[L=2](2){2'} \SO[L=3](3){3'} \SO[L=4](4){4'} \SO[L=5](5){5'} \SO[L=6](6){6'}
                    \onslide<4->{\Edge(1.south)(4'.north)}
                    \onslide<5->{\Edge(4.south)(5'.north)}
                    \onslide<6->{\Edge(5.south)(1'.north)}
                    \onslide<7->{\Edge(2.south)(6'.north)}
                    \onslide<8->{\Edge(6.south)(2'.north)}
                    \onslide<9->{\Edge(3.south)(3'.north)}
                    \tkzDefPoint(-0.5,-0.5){lab}\tkzLabelPoint[left](lab){$\sigma$}
                \end{tikzpicture}
            \end{center}

            It means
            $$\onslide<4->{1^\sigma = 4,\ \onslide<5->{4^\sigma = 5,\ \onslide<6->{5^\sigma = 1,\ \onslide<7->{2^\sigma = 6,\ \onslide<8->{6^\sigma = 2,\ \onslide<9->{3^\sigma = 3.}}}}}}$$}}
\end{slide}

\begin{slide}
    \textit{Product/composition:} for $\sigma,\tau \in \Sym(n)$, $\sigma\tau$ means ``first $\sigma$, then $\tau$'', so $i^{\sigma\tau} = (i^\sigma)^\tau$. \onslide<2->{E.g. $\sigma = (1,2,3),\onslide<3->{\tau = (1,3)(2,4) \in \Sym(4)$,}

    \begin{center}
        \begin{tikzpicture}[x=1cm,y=1.5cm]
            \GraphInit[vstyle=Normal]
            \tikzset{VertexStyle/.style={draw=none}}
            \tikzset{EdgeStyle/.style={->}}
            \Vertex{1} \EA(1){2} \EA(2){3} \EA(3){4}
            \SO[L=1](1){1'} \SO[L=2](2){2'} \SO[L=3](3){3'} \SO[L=4](4){4'}
            \Edge(1.south)(2'.north) \Edge(2.south)(3'.north) \Edge(3.south)(1'.north) \Edge(4.south)(4'.north)
            \tkzDefPoint(-0.5,-0.5){lab}\tkzLabelPoint[left](lab){$\sigma$}
            \onslide<3->{\SO[L=1](1'){1''} \SO[L=2](2'){2''} \SO[L=3](3'){3''} \SO[L=4](4'){4''}
                \Edge(1'.south)(3''.north) \Edge(3'.south)(1''.north) \Edge(2'.south)(4''.north) \Edge(4'.south)(2''.north)
                \tkzDefPoint(-0.5,-1.5){lab2}\tkzLabelPoint[left](lab2){$\tau$}
            }
            \onslide<4->{
                \EA[unit=3,L=1](4){c1} \EA[L=2](c1){c2} \EA[L=3](c2){c3} \EA[L=4](c3){c4}
                \EA[unit=3,L=1](4''){c1'} \EA[L=2](c1'){c2'} \EA[L=3](c2'){c3'} \EA[L=4](c3'){c4'}
                \onslide<5->{\Edge(c1.south)(c4'.north)}
                \onslide<6->{\Edge(c4.south)(c2'.north)}
                \onslide<7->{\Edge(c2.south)(c1'.north)}
                \onslide<8->{\Edge(c3.south)(c3'.north)}
                \tkzDefPoint(5.5,-1){lab3}\tkzLabelPoint[left](lab3){$\sigma\tau$}
            }
        \end{tikzpicture}
    \end{center}
    \onslide<4->{$$\sigma\tau = (1,2,3)(1,3)(2,4) = (1,\onslide<5->{4,\onslide<6->{2\onslide<7->{)\onslide<8->{\in \Sym(4).}}}}$$}}

    \vspace{-0.5cm}
    \textit{Note:} here, $\sigma\tau \neq \tau\sigma$, since $1^{\sigma\tau} = 4$ but $1^{\tau\sigma} = (1^\tau)^\sigma = 3^\sigma = 1$. Identity $1 = ()$ satisfies $1 \sigma = \sigma 1 = \sigma$ for $\sigma \in \Sym(n)$.
\end{slide}

\subsection{Permutation groups}

\begin{slide}
    \textit{Note:} for $\sigma,\tau,\pi \in \Sym(n)$, (i) $\sigma\tau \in \Sym(n)$, (ii) $1 = () \in \Sym(n)$, (iii) $\sigma^{-1} \in \Sym(n)$, (iv) $(\sigma\tau)\pi = \sigma(\tau\pi)$. If true for subset:

    \begin{definition}[permutation group]
        \vspace{0pt}
        \textbf{Permutation group} of degree $n$ is subset $G \leq \Sym(n)$ satisfying:
        \begin{enumerate}[(i)]
            \item \textbf{(closure)} $\sigma\tau \in G$ for $\sigma,\tau \in G$; \pause
            \item \textbf{(identity)} $1 = () \in G$; \pause
            \item \textbf{(inverses)} $\sigma^{-1} \in G$ for $\sigma \in G$.
        \end{enumerate}
    \end{definition}

    \begin{example}[alternating group]
        \vspace{0pt}
        \textbf{Alternating group} $\Alt(3) = \{(),(1,2,3),(1,3,2)\} < \Sym(3)$. \\
        In general, $\Alt(n)$ is all \textit{even} permutations of $[n]$ (product of even \# of \textit{transpositions} $(i,j)$, e.g. $(1,2,3) = (1,2)(1,3)$).
    \end{example}
\end{slide}

% \subsection{Order of permutations}

% \begin{slide}
%     \begin{definition}[order]
%         \vspace{0pt}
%         \textbf{Order} of $\sigma \in G$ is least $k \in \Z_+$ with $\sigma^k = \sigma \dotsb \sigma = 1$.
%     \end{definition} \pause

%     \begin{example}
%         \vspace{0pt}
%         Consider $\sigma = (1,4,5)(2,6) \in \Sym(6)$.
%         \begin{center}
%             \begin{tikzpicture}[x=0.6cm,y=0.6cm]
%                 \GraphInit[vstyle=Normal]
%                 \tikzset{VertexStyle/.style={draw=none}}
%                 \Vertices{circle}{5,1,4}
%                 \Vertex[x=3,y=0]{2}
%                 \Vertex[x=5,y=0]{6}
%                 \Vertex[x=7,y=0]{3}
%                 \Loop[dist=1cm,dir=EA,style={thick,->}](3)
%                 \tikzset{EdgeStyle/.style={->,bend right}}
%                 \Edges(1,4,5,1)
%                 \Edges[style={bend right=60}](2,6,2)
%                 \tkzDefPoint(-3,0){lab}\tkzLabelPoint[centered](lab){$\sigma$}
%             \end{tikzpicture}
%         \end{center} \pause
%         Then $1^{\sigma^3} = 4^{\sigma^2} = 5^\sigma = 1$, \pause $4^{\sigma^3} = 4$, $5^{\sigma^3} = 5$, $2^{\sigma^2} = 2$, $6^{\sigma^2} = 6$ so \pause $\sigma^6 = () = 1$; order of $\sigma$ is 6. \pause
%     \end{example}

%     \begin{proposition}
%         Order of $\sigma \in \Sym(n)$ is lcm of cycle lengths.
%     \end{proposition}
% \end{slide}

\subsection{Generating a group}

\begin{slide}
    \begin{definition}[generator]
        \vspace{0pt}
        Set $X$ \textbf{generates} $G$ if every $\sigma \in G$ is $\sigma = x_1^{\pm 1} \dotsb x_r^{\pm 1}$ for some $r \in \N$, $x_i \in X$ \textbf{generators}; write $G = \langle X \rangle$. \pause

        (If $G = \langle X \rangle$ for some $X$ with $|X| = 1$, $G$ is \textbf{cyclic}.)
    \end{definition} \pause

    \begin{example}[cyclic group]
        \vspace{0pt}
        Consider $\Alt(3) = \{(),(1,2,3),(1,3,2)\}$: $(1,2,3)^2 = (1,3,2)$, $(1,2,3)^3 = ()$, so $\Alt(3) = \langle(1,2,3)\rangle$ is cyclic (only for $n = 3$).
    \end{example}

    \begin{example}[symmetric group]
        \vspace{0pt}
        Consider $\Sym(3) = \{(),(1,2),(1,3),(2,3),(1,2,3),(1,3,2)\}$. \\
        Not cyclic, but $\Sym(3) = \langle(1,2),(2,3)\rangle$ (adjacent swaps). \\
        Also, $\Sym(3) = \langle(1,2),(1,2,3)\rangle$, e.g. $(2,3) = (1,2,3)(1,2)$.
    \end{example}
\end{slide}

\subsection{Group actions}

\begin{slide}
    \begin{definition}[group action]
        \vspace{0pt}
        For permutation group $G$ and set $\Omega \neq \emptyset$, \textbf{$G$-action} is map $\Omega \times G \to \Omega$, $(\alpha,\sigma) \mapsto \alpha^\sigma$ s.t. $\alpha^1 = \alpha$ and $\alpha^{\sigma\tau} = (\alpha^\sigma)^\tau$ for $\alpha \in \Omega$ and $\sigma,\tau \in G$.
    \end{definition}

    \textit{Idea:} $\alpha \in \Omega$ is \textit{state}, apply \textit{move} $\sigma \in G$ to get state $\alpha^\sigma \in \Omega$, in way that respects permutation product. \pause

    \begin{example}[natural action]
        \vspace{0pt}
        $G \leq \Sym(n)$ acts on $\Omega = [n]$ by $\alpha^\sigma := \alpha^\sigma$ (image) for $\alpha \in [n]$, $\sigma \in G$.
    \end{example} \pause

    \begin{example}[right regular action]
        \vspace{0pt}
        Perm group $G$ acts on $\Omega = G$ (itself) via $\alpha^\sigma := \alpha\sigma$ for $\alpha,\sigma \in G$. (\textit{Check:} $\alpha^1 = \alpha 1 = \alpha$ and $\alpha^{\sigma\tau} = \alpha(\sigma\tau) = (\alpha\sigma)\tau = (\alpha^\sigma)^\tau$.)
    \end{example}
\end{slide}

\begin{slide}
    \begin{example}[dihedral group]
        \vspace{0pt}
        Let $r = (1,2,3,4),s = (1,4)(2,3) \in \Sym(4)$. \textbf{Dihedral group} is $D_8 := \langle r,s \rangle = \{1,r,r^2,r^3,s,sr,sr^2,sr^3\}$, \textit{``symmetries of square''}.

        \onslide<2->{\textit{Note:} $sr = (2,4)$, $sr^2 = (1,2)(3,4)$. Action of $D_8$ on vertices of square (labelled by $[4]$): $\sigma \in D_8$ sends vertex at $i$ to $i^\sigma$.}

        \begin{center}
            \begin{tikzpicture}[x=1cm,y=1cm]
                \GraphInit[vstyle=Classic]
                \tikzset{VertexStyle/.append style={minimum size=1pt}}
                % TOP LEFT
                \tkzDefPoint(0.5,0.25){l2}
                \tkzDefPoint(0.5,-1.25){l3}
                \tkzDefPoint(-0.25,0.25){l4}
                \tkzDefPoint(1.25,-1.25){l5}
                \tikzset{EdgeStyle/.style={dotted}}
                \onslide<5->{\Edge(l2)(l3)} % s
                \onslide<7->{\Edge[color=metrogreen](l4)(l5)} % sr
                \tikzset{EdgeStyle/.style={-}}
                \onslide<3->{{\Vertex[Lpos=180]{1} \SO[Lpos=180](1){2} \EA(2){3} \NO(3){4}
                            \AddVertexColor{red}{1}
                            \AddVertexColor{green}{2}
                            \AddVertexColor{blue}{3}
                            \AddVertexColor{yellow}{4}}
                    \Edges(1,2,3,4,1)
                    \tikzset{EdgeStyle/.style={->}}
                    \tkzDefPoint(1.5,-0.5){a1}
                    \tkzDefPoint(3.5,-0.5){a2}
                    \onslide<4->{\Edge[label=$r$,labelcolor=metropolis](a1)(a2)
                        \tkzDefPoint(2.5,0){l1}\tkzLabelPoint[centered](l1){$\curvearrowleft$}
                        % TOP RIGHT
                        \tkzDefPoint(3.75,-0.5){l2}
                        \tkzDefPoint(5.25,-0.5){l3}
                        \tikzset{EdgeStyle/.style={dotted}}
                        \onslide<8->{\Edge(l2)(l3)} % sr^2
                        \tikzset{EdgeStyle/.style={-}}
                        \onslide<4->{\Vertex[x=4,y=0,Lpos=180]{1} \SO[Lpos=180](1){2} \EA(2){3} \NO(3){4}
                            \AddVertexColor{yellow}{1}
                            \AddVertexColor{red}{2}
                            \AddVertexColor{green}{3}
                            \AddVertexColor{blue}{4}
                            \Edges(1,2,3,4,1)}
                        \tikzset{EdgeStyle/.style={->}}
                        \tkzDefPoint(0.5,-1.5){a1}
                        \tkzDefPoint(0.5,-2.5){a2}
                        \onslide<5->{\Edge[label=$s$,labelcolor=metropolis](a1)(a2)}
                        % BOTTOM LEFT
                        \tikzset{EdgeStyle/.style={-}}
                        \onslide<5->{\Vertex[x=0,y=-3,Lpos=180]{1} \SO[Lpos=180](1){2} \EA(2){3} \NO(3){4}
                            \AddVertexColor{yellow}{1}
                            \AddVertexColor{blue}{2}
                            \AddVertexColor{green}{3}
                            \AddVertexColor{red}{4}
                            \Edges(1,2,3,4,1)}
                        \tikzset{EdgeStyle/.style={->}}
                        \tkzDefPoint(1.5,-3.5){a1}
                        \tkzDefPoint(3.5,-3.5){a2}
                        \onslide<6->{\Edge[label=$r$,labelcolor=metropolis](a1)(a2)
                            \tkzDefPoint(2.5,-3){l1}\tkzLabelPoint[centered](l1){$\curvearrowleft$}}
                        % BOTTOM RIGHT
                        \tikzset{EdgeStyle/.style={-}}
                        \onslide<6->{\Vertex[x=4,y=-3,Lpos=180]{1} \SO[Lpos=180](1){2} \EA(2){3} \NO(3){4}
                            \AddVertexColor{red}{1}
                            \AddVertexColor{yellow}{2}
                            \AddVertexColor{blue}{3}
                            \AddVertexColor{green}{4}
                            \Edges(1,2,3,4,1)}
                        \tikzset{EdgeStyle/.style={->}}
                        \tkzDefPoint(4.5,-1.5){a1}
                        \tkzDefPoint(4.5,-2.5){a2}
                        \onslide<8->{\Edge[label=$sr^2$,labelcolor=metropolis](a1)(a2)}
                        \tkzDefPoint(1.5,-1.5){a1}
                        \tkzDefPoint(3.5,-2.5){a2}
                        \onslide<7->{\Edge[label=$sr$,labelcolor=metropolis,labeltext=metrogreen,color=metrogreen](a1)(a2)}}}
            \end{tikzpicture}
        \end{center}
    \end{example}
\end{slide}

\subsection{Orbits and stabilisers}

\begin{slide}
    \begin{definition}[orbit]
        \vspace{0pt}
        If $G$ acts on $\Omega$, then \textbf{orbit} of $\alpha \in \Omega$ is $\alpha^G := \{\alpha^\sigma : \sigma \in G\}$. \\
        \textit{Idea:} states $\alpha^\sigma \in \Omega$ reachable from fixed $\alpha \in \Omega$ by moves $\sigma \in G$. \pause
    \end{definition}

    \begin{definition}[stabiliser]
        \vspace{0pt}
        If $G$ acts on $\Omega$, then \textbf{stabiliser} of $\alpha \in \Omega$ is $G_\alpha := \{\sigma \in G : \alpha^\sigma = \alpha\}$. \\
        \textit{Idea:} moves $\sigma \in G$ that fix given $\alpha \in \Omega$. \pause
    \end{definition}

    \begin{example}[right regular action]
        \vspace{0pt}
        $G$ acts on $\Omega = G$ via $\alpha^\sigma = \alpha \sigma$ for $\alpha,\sigma \in G$. Orbit of $\alpha \in G$ is \pause $\Omega = G$ ($\alpha^{\alpha^{-1}\beta} = \beta \in G$); stabiliser of $\alpha$ is \pause $\{1\} = 1$ ($\alpha\sigma = \alpha \implies \sigma = 1$).

        One orbit only: \textbf{transitive} action.
    \end{example}
\end{slide}

\begin{slide}
    Orbit $\alpha^G$: states $\alpha^\sigma \in \Omega$ reachable from fixed $\alpha$ by moves $\sigma \in G$. \\
    Stabiliser $G_\alpha$: moves $\sigma \in G$ that fix given $\alpha$.

    \begin{example}[dihedral group]
        \vspace{0pt}
        Recall $G = D_8 = \langle r,s \rangle = \{1,r,r^2,r^3,s,sr,sr^2,sr^3\} \leq \Sym(4)$ where $r = (1,2,3,4)$, $s = (1,4)(2,3)$.

        Orbit of 1: $1^1 = 1$, $1^r = 2$, $1^{r^2} = 3$, $1^{r^3} = 4$, so $1^G = [4]$ (transitive).

        Stabiliser of 1: $sr = (2,4)$, $sr^2 = (1,2)(3,4)$, $sr^3 = (1,3)$, so $G_1 = \{(),(2,4)\} = \{1,sr\}$.

        \textit{Note:} $|1^G||G_1| = 4 \cdot 2 = 8 = |G|$. Coincidence?
    \end{example}

    \begin{theorem}[orbit-stabiliser]
        \vspace{0pt}
        If $G$ acts on $\Omega$, then for $\alpha \in \Omega$, $|\alpha^G||G_\alpha| = |G|$.
    \end{theorem}
\end{slide}