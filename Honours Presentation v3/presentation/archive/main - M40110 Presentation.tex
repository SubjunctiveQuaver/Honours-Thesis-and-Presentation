% \documentclass{beamer}
\documentclass[handout]{beamer}

% Settings and commands
\usepackage[utf8]{inputenc} % Accept different input encodings
\usepackage{mathtools} % Extra symbols and fixes to amsmath
\usepackage{thmtools} % Theorem tools
\usepackage{graphicx}
% \usepackage[shortlabels]{enumitem} % Customising lists - incompatible with beamer
\usepackage{centernot} % Centred negation slash
\usepackage{hyperref}
\usepackage{tcolorbox} % Boxes surrounding text
\usepackage{tikz-cd} % Commutative diagrams
\usepackage{newtx} % Nicer font
\usepackage{bookmark} % Hyperlinks and bookmarks
\usepackage{xcolor}
\usepackage{tkz-base}
\usepackage{tkz-graph}
\usepackage{tkz-euclide}
\usepackage{tikzscale}
\usepackage{listings}
\usepackage{csquotes}
\usepackage{accsupp}
\newcommand{\emptyaccsupp}[1]{\BeginAccSupp{ActualText={}}#1\EndAccSupp{}}

\usetheme[progressbar=frametitle]{metropolis}
\usefonttheme{professionalfonts}
\newcounter{insub}
\AtBeginSubsection[]{\setcounter{insub}{0}}
\AtBeginEnvironment{slide}{\addtocounter{insub}{1}}
\newenvironment<>{slide}{\begin{frame}#1%
        \frametitle{\subsecname{}\ifthenelse{\theinsub=1}{}{\enskip(\roman{insub})}}
        }{\end{frame}}
% \setbeamertemplate{theorems}[numbered]
\definecolor{metropolis}{HTML}{FAFAFA}

% ---------------------------------------- %
% SETTINGS

\tcbuselibrary{breakable}
\newtcolorbox{framedbox}{standard jigsaw, opacityback=0, boxrule=1pt, breakable, before skip=8pt, after skip=8pt}

% \let \originalLeft \left % Fixing \left and \right
% \let \originalRight \right
% \renewcommand{\left}{\mathopen{} \mathclose \bgroup \originalLeft}
% \renewcommand{\right}{\aftergroup \egroup \originalRight}

\newtheorem{proposition}{Proposition}

%% tkz-graph
% \GraphInit[vstyle=Hasse]
% \GraphInit[vstyle=Welsh]
% \GraphInit[vstyle=Normal]
% \GraphInit[vstyle=Classic]
\SetGraphUnit{1}
\SetVertexMath
\tikzset{VertexStyle/.append style={minimum size=8pt}}

%% listings
% Code block styles (listings)
\definecolor{codegreen}{rgb}{0,0.6,0}
\definecolor{codegray}{rgb}{0.5,0.5,0.5}
\definecolor{codepurple}{rgb}{0.58,0,0.82}
\definecolor{backcolour}{rgb}{0.95,0.95,0.92}
\lstdefinestyle{default}{
    basewidth={.5em,0.5em},
    basicstyle=\ttfamily,
    breakatwhitespace=false,
    breaklines=true,
    captionpos=b,
    xleftmargin=15pt,
    framexleftmargin=2pt,
    framexrightmargin=2pt,
    keepspaces=true,
    numbers=left,
    numbersep=5pt,
    tabsize=4,
    columns=fullflexible,
    backgroundcolor=\color{backcolour},
    commentstyle=\itshape\color{codegreen},
    keywordstyle=\color{magenta},
    numberstyle=\footnotesize\itshape\color{black}\emptyaccsupp,
    stringstyle=\color{codepurple}
}
% \lstdefinelanguage{GAP}{%
%     morekeywords={%
%             Assert,Info,IsBound,QUIT,%
%             TryNextMethod,Unbind,and,break,%
%             continue,do,elif,%
%             else,end,false,fi,for,%
%             function,if,in,local,%
%             mod,not,od,or,%
%             quit,rec,repeat,return,%
%             then,true,until,while%
%         },%
%     sensitive,%
%     morecomment=[l]\#,%
%     morestring=[b]",%
%     morestring=[b]',%
% }[keywords,comments,strings]
\lstset{style=default,language=GAP}
\input{commands.tex}

\title{\textbf{Computing in permutation groups}}
\author{\textbf{Lawrence Chen}}
\institute{\textbf{M40110 Honours presentation}}
\date{\today}

\begin{document}

\begin{frame}[plain, noframenumbering]
    \titlepage
\end{frame}

\begin{frame}[plain, noframenumbering]{Contents}
    \tableofcontents
\end{frame}

\section{Introduction -- computational group theory}

% \subsection{Computational group theory}

% \begin{slide}
%     Lol
% \end{slide}

\subsection{Permutations and graph automorphisms}

\begin{slide}
    Consider the \textbf{permutation} $\sigma = (1,6,2,7,3,8)(4,5) \in \Sym(8)$. Instead of $\sigma(n)$, write $n^\sigma$: so $1^\sigma = 6$, $2^\sigma = 7$, $4^\sigma = 5$, etc. \pause

    Let $\Gamma$ be the graph below with vertex set $V = \{1,\dotsc,8\}$:

    \begin{center}
        \begin{tikzpicture}
            \GraphInit
            \Vertex[x=-1, y=1, Lpos=180]{1}
            \Vertex[x=-1, y=0, Lpos=180]{2}
            \Vertex[x=-1, y=-1, Lpos=180]{3}
            \Vertex[x=0, y=0, Lpos=-90]{4}
            \Vertex[x=1, y=0, Lpos=-90]{5}
            \Vertex[x=2, y=1]{6}
            \Vertex[x=2, y=0]{7}
            \Vertex[x=2, y=-1]{8}
            \Edges(1,4,5,6)
            \Edges(2,4,3)
            \Edges(7,5,8)
            \node at (0.5,0.5){$\Gamma$};
        \end{tikzpicture} \pause \quad $\underset{\sigma^{-1}}{\overset{\sigma}{\rightleftarrows}}$ \quad
        \begin{tikzpicture}
            \GraphInit
            \Vertex[x=-1, y=1, Lpos=180]{6}
            \Vertex[x=-1, y=0, Lpos=180]{7}
            \Vertex[x=-1, y=-1, Lpos=180]{8}
            \Vertex[x=0, y=0, Lpos=-90]{5}
            \Vertex[x=1, y=0, Lpos=-90]{4}
            \Vertex[x=2, y=1]{2}
            \Vertex[x=2, y=0]{3}
            \Vertex[x=2, y=-1]{1}
            \Edges(1,4,5,6)
            \Edges(2,4,3)
            \Edges(7,5,8)
            \node at (0.5,0.5){$\Gamma$};
        \end{tikzpicture} \pause
    \end{center}

    % Applying $\sigma$ to the vertices of $\Gamma$, we get the right relabelling of $\Gamma$ above, which preserves edges. The map $\sigma$ is a \textbf{graph automorphism}.

    The map $\sigma$ is a \textbf{graph automorphism}. Recall $1 = () \in \Sym(8)$.

    Automorphisms of $\Gamma$ form a \textbf{(permutation) group} $G = \Aut(\Gamma)$. Read $gh$ as ``first $g$, then $h$''. Here, $\sigma^{-1} = (1,8,3,7,2,6)(4,5)$. % If $g,h \in G$, then $gh,g^{-1} \in G$, where $gh$ is ``first $g$, then $h$'', and $g^{-1}$ satisfies $gg^{-1} = g^{-1}g = 1$. 
\end{slide}

\begin{slide}
    Consider again $\sigma = (1,6,2,7,3,8)(4,5) \in G$. Let $\tau = (2,3)(6,7,8) \in G$. Then $\sigma\tau = (1,7,2,8)(3,6)(4,5) \in G$: \pause

    \begin{center}
        \begin{tikzpicture}[scale=0.67]
            \GraphInit
            \Vertex[x=-1, y=1, Lpos=180]{1}
            \Vertex[x=-1, y=0, Lpos=180]{2}
            \Vertex[x=-1, y=-1, Lpos=180]{3}
            \Vertex[x=0, y=0, Lpos=-90]{4}
            \Vertex[x=1, y=0, Lpos=-90]{5}
            \Vertex[x=2, y=1]{6}
            \Vertex[x=2, y=0]{7}
            \Vertex[x=2, y=-1]{8}
            \Edges(1,4,5,6)
            \Edges(2,4,3)
            \Edges(7,5,8)
            \node at (0.5,0.5){$\Gamma$};
        \end{tikzpicture} $\overset{\sigma}{\to}$
        \begin{tikzpicture}[scale=0.67]
            \GraphInit
            \Vertex[x=-1, y=1, Lpos=180]{6}
            \Vertex[x=-1, y=0, Lpos=180]{7}
            \Vertex[x=-1, y=-1, Lpos=180]{8}
            \Vertex[x=0, y=0, Lpos=-90]{5}
            \Vertex[x=1, y=0, Lpos=-90]{4}
            \Vertex[x=2, y=1]{2}
            \Vertex[x=2, y=0]{3}
            \Vertex[x=2, y=-1]{1}
            \Edges(1,4,5,6)
            \Edges(2,4,3)
            \Edges(7,5,8)
            \node at (0.5,0.5){$\Gamma$};
        \end{tikzpicture} $\overset{\tau}{\to}$ \pause
        \begin{tikzpicture}[scale=0.67]
            \GraphInit
            \Vertex[x=-1, y=1, Lpos=180]{7}
            \Vertex[x=-1, y=0, Lpos=180]{8}
            \Vertex[x=-1, y=-1, Lpos=180]{6}
            \Vertex[x=0, y=0, Lpos=-90]{5}
            \Vertex[x=1, y=0, Lpos=-90]{4}
            \Vertex[x=2, y=1]{3}
            \Vertex[x=2, y=0]{2}
            \Vertex[x=2, y=-1]{1}
            \Edges(1,4,5,6)
            \Edges(2,4,3)
            \Edges(7,5,8)
            \node at (0.5,0.5){$\Gamma$};
        \end{tikzpicture} \pause
    \end{center}

    \begin{center}
        \begin{tikzpicture}
            \GraphInit
            \Vertex[x=-1, y=1, Lpos=180]{1}
            \Vertex[x=-1, y=0, Lpos=180]{2}
            \Vertex[x=-1, y=-1, Lpos=180]{3}
            \Vertex[x=0, y=0, Lpos=-90]{4}
            \Vertex[x=1, y=0, Lpos=-90]{5}
            \Vertex[x=2, y=1]{6}
            \Vertex[x=2, y=0]{7}
            \Vertex[x=2, y=-1]{8}
            \Edges(1,4,5,6)
            \Edges(2,4,3)
            \Edges(7,5,8)
            \node at (0.5,0.5){$\Gamma$};
        \end{tikzpicture} \quad $\overset{\sigma\tau}{\to}$ \quad \pause
        \begin{tikzpicture}
            \GraphInit
            \Vertex[x=-1, y=1, Lpos=180]{7}
            \Vertex[x=-1, y=0, Lpos=180]{8}
            \Vertex[x=-1, y=-1, Lpos=180]{6}
            \Vertex[x=0, y=0, Lpos=-90]{5}
            \Vertex[x=1, y=0, Lpos=-90]{4}
            \Vertex[x=2, y=1]{3}
            \Vertex[x=2, y=0]{2}
            \Vertex[x=2, y=-1]{1}
            \Edges(1,4,5,6)
            \Edges(2,4,3)
            \Edges(7,5,8)
            \node at (0.5,0.5){$\Gamma$};
        \end{tikzpicture}
    \end{center}

    We say $G = \Aut(\Gamma)$ \textbf{acts (naturally)} on $\Omega = V$.

    % Instead of $\tau(v)$ for $v \in V$, write $v^\tau$, e.g. $1^\tau = 1$, $2^\tau = 3$, and $6^\tau = 7$.

    % Applying $1 = ()$ to the vertices of $\Gamma$ clearly results in $\Gamma$, so $1 \in G$. % Each $g \in G$ permutes the vertices in $\Gamma$ such that the resulting graph is stil $\Gamma$.
\end{slide}

\subsection{Group actions and the OST}

\begin{slide}
    The \textbf{orbit} of $1 \in V$ is $1^G = \{1^g : g \in G\} = \{1,2,3,6,7,8\}$ and the \textbf{stabiliser} of $1 \in V$ is $\Stab_G(1) = \{g \in G : 1^g = 1\}$, which contains $\tau = (2,3)(6,7,8)$: % , with $v^g \in V$ for $v \in V$ and $g \in G$. Then $v^{gh} = (v^g)^h \in V$ if $h \in G$ also; $1 \in G$ acts \textit{trivially}. \pause

    % An \textbf{orbit} is $1^G = \{1^g : g \in G\} = \{1,2,3,6,7,8\}$ and a \textbf{stabiliser} is $\Stab_G(1) = \{g \in G : 1^g = 1\}$, which contains $\tau = (2,3)(6,7,8)$.

    \begin{center}
        \begin{tikzpicture}
            \GraphInit
            \Vertex[x=-1, y=1, Lpos=180]{1}
            \Vertex[x=-1, y=0, Lpos=180]{2}
            \Vertex[x=-1, y=-1, Lpos=180]{3}
            \Vertex[x=0, y=0, Lpos=-90]{4}
            \Vertex[x=1, y=0, Lpos=-90]{5}
            \Vertex[x=2, y=1]{6}
            \Vertex[x=2, y=0]{7}
            \Vertex[x=2, y=-1]{8}
            \Edges(1,4,5,6)
            \Edges(2,4,3)
            \Edges(7,5,8)
            \node at (0.5,0.5){$\Gamma$};
        \end{tikzpicture} \quad $\overset{\tau}{\to}$ \quad
        \begin{tikzpicture}
            \GraphInit
            \Vertex[x=-1, y=1, Lpos=180]{1}
            \Vertex[x=-1, y=0, Lpos=180]{3}
            \Vertex[x=-1, y=-1, Lpos=180]{2}
            \Vertex[x=0, y=0, Lpos=-90]{4}
            \Vertex[x=1, y=0, Lpos=-90]{5}
            \Vertex[x=2, y=1]{7}
            \Vertex[x=2, y=0]{8}
            \Vertex[x=2, y=-1]{6}
            \Edges(1,4,5,6)
            \Edges(2,4,3)
            \Edges(7,5,8)
            \node at (0.5,0.5){$\Gamma$};
        \end{tikzpicture} \pause
    \end{center}

    The \textbf{orbit-stabiliser theorem (OST)} says that $|G| = |\alpha^G||\Stab_G(\alpha)|$ for $\alpha \in \Omega$, if $G$ \textit{acts} on $\Omega$. But what is $|G| = |\Aut(\Gamma)|$ here?
\end{slide}

% \begin{slide}
%     We consider a stabiliser chain for $G$ (and find $|G|$ and transversals:
%     \begin{itemize}
%         \item Let $G_1 = \Stab_{G_0}(4)$. Then $4^{G_0} = \{4,5\}$, and take $T_1 = \{(),(1,6)(2,7)(3,8)(4,5)\}$.
%         \item Let $G_2 = \Stab_{G_1}(1) = \Stab_G(4,1)$. Then $1^{G_1} = \{1,2,3\}$, and take $T_1 = \{(),(1,2),(1,3)\}$.
%         \item Let $G_3 = \Stab_{G_2}(2) = \Stab_G(4,1,2)$. Then $2^{G_2} = \{2,3\}$, and take $T_2 = \{(),(2,3)\}$.
%         \item Let $G_4 = \Stab_{G_3}(6) = \Stab_G(4,1,2,6)$. Then $6^{G_3} = \{6,7,8\}$, and take $T_3 = \{(),(6,7),(6,8)\}$.
%         \item Let $G_5 = \Stab_{G_4}(7) = \Stab_G(4,1,2,6,7) = 1$. Then $7^{G_4} = \{7,8\}$, and take $T_4 = \{(),(7,8)\}$.
%     \end{itemize}
%     So we see that $B = [4,1,2,6,7]$ is a base for $G$ with stabiliser chain $G = G_0 > G_1 > G_2 > G_3 > G_4 > G_5 = 1$ and associated transversals $T_1,\dotsc,T_5$. Moreover, from \autoref{prop:stabiliser_chain_indexes}, we see that $|G| = |T_1| \dotsb |T_5| = 2 \cdot 3 \cdot 2 \cdot 3 \cdot 2 = 72$, so there are 72 automorphisms (relabellings) of $\Gamma$. From \autoref{lem:transversal_gives_bsgs}, $\{(1,6)(2,7)(3,8)(4,5),(1,2),(1,3),(2,3),(6,7),(6,8),(7,8)\}$ is a strong generating set $S$ for $G$ relative to $B$ with size $\sum_i |T_i| - 5 = 7$.
% \end{slide}

\section{Computing with permutation groups}

\subsection{Bases and stabiliser chains}

\begin{slide}
    A \textbf{base} for $G$ is $B = [\beta_1,\dotsc,\beta_r] \subseteq \Omega$ such that only $1 \in G$ fixes all of $B$. Its \textbf{stabiliser chain} is $G = G_0 \geq G_1 \geq \dotsb \geq G_r = \{1\}$, where $G_i = \Stab_{G_{i - 1}}(\beta_i) = \Stab_G(\beta_1,\dotsc,\beta_i)$. \pause

    We can find \textbf{transversals} $T_1,\dotsc,T_r$ with each $1 \ni T_i \subseteq G_{i - 1}$ and $|T_i| = |\beta_i^{G_{i - 1}}|$, such that $g \in G$ is uniquely $g = t_r t_{r - 1} \dotsb t_1$ with $t_i \in T_i$. \pause

    \begin{center}
        \begin{tikzpicture}
            \GraphInit
            \Vertex[x=-1, y=1, Lpos=180]{1}
            \Vertex[x=-1, y=0, Lpos=180]{2}
            \Vertex[x=-1, y=-1, Lpos=180]{3}
            \Vertex[x=0, y=0, Lpos=-90]{4}
            \Vertex[x=1, y=0, Lpos=-90]{5}
            \Vertex[x=2, y=1]{6}
            \Vertex[x=2, y=0]{7}
            \Vertex[x=2, y=-1]{8}
            \Edges(1,4,5,6)
            \Edges(2,4,3)
            \Edges(7,5,8)
            \node at (0.5,0.5){$\Gamma$};
        \end{tikzpicture}
    \end{center}

    A base for $G = \Aut(\Gamma)$ is $B = [4,1,2,6,7]$. The stabiliser chain is $G = G_0 \geq G_1 \geq G_2 \geq G_3 \geq G_4 \geq G_5 = \{1\}$. (Can use \texttt{GAP}.)
\end{slide}

\subsection{The size of the group}

\begin{slide}
    Let $T_1,\dotsc,T_r$ be transversals. Then the OST says $|G| = |T_1| \dotsb |T_r|$. \pause Recall that $G_0 = G$, $G_i = \Stab_G(\beta_1,\dotsc,\beta_i)$ and $B = [4,1,2,6,7]$.

    \begin{center}
        \begin{tikzpicture}[scale=0.75]
            \GraphInit
            \Vertex[x=-1, y=1, Lpos=180]{1}
            \Vertex[x=-1, y=0, Lpos=180]{2}
            \Vertex[x=-1, y=-1, Lpos=180]{3}
            \Vertex[x=0, y=0, Lpos=-90]{4}
            \Vertex[x=1, y=0, Lpos=-90]{5}
            \Vertex[x=2, y=1]{6}
            \Vertex[x=2, y=0]{7}
            \Vertex[x=2, y=-1]{8}
            \Edges(1,4,5,6)
            \Edges(2,4,3)
            \Edges(7,5,8)
            \node at (0.5,0.5){$\Gamma$};
        \end{tikzpicture}
    \end{center}
    $$\begin{array}{c|c|c|c}
            G_i & \beta_i & \beta_i^{G_{i - 1}} & T_i \subseteq G_{i - 1}                              \\ \hline
            G_1 & 4       & \{4,5\}             & \{(),(1,6)(2,7)(3,8)(4,\textcolor{red}{5})\}         \\ \pause
            G_2 & 1       & \{1,2,3\}           & \{(),(1,\textcolor{red}{2}),(1,\textcolor{red}{3})\} \\ \pause
            G_3 & 2       & \{2,3\}             & \{(),(2,\textcolor{red}{3})\}                        \\ \pause
            G_4 & 6       & \{6,7,8\}           & \{(),(6,\textcolor{red}{7}),(6,\textcolor{red}{8})\} \\ \pause
            G_5 & 7       & \{7,8\}             & \{(),(7,\textcolor{red}{8})\} \pause
        \end{array}$$

    So $|G| = |\Aut(\Gamma)| = 2 \cdot 3 \cdot 2 \cdot 3 \cdot 2 = 72$ and $|\Stab_G(1)| = 72/6 = 12$.
\end{slide}

\subsection{Generating random elements}

\begin{slide}
    We can get a random element $g = t_r t_{r - 1} \dotsb t_1 \in G$ where each $t_i \in T_i$. \pause
    $$\begin{array}{c|c|c}
            i & T_i \subseteq G_{i - 1}                      & g_i = \textcolor{red}{t_i} g_{i - 1} \\ \hline
            1 & \{(),\textcolor{red}{(1,6)(2,7)(3,8)(4,5)}\} & (1,6)(2,7)(3,8)(4,5)                 \\ \pause
            2 & \{(),\textcolor{red}{(1,2)},(1,3)\}          & (1,7,2,6)(3,8)(4,5)                  \\ \pause
            3 & \{(),\textcolor{red}{(2,3)}\}                & (1,7,2,8,3,6)(4,5)                   \\ \pause
            4 & \{\textcolor{red}{()},(6,7),(6,8)\}          & (1,7,2,8,3,6)(4,5)                   \\ \pause
            5 & \{(),\textcolor{red}{(7,8)}\}                & (1,7,3,6)(2,8)(4,5)  \pause
        \end{array}$$

    Our random automorphism of $\Gamma$ is $g = (1,7,3,6)(2,8)(4,5)$: \pause

    \begin{center}
        \begin{tikzpicture}
            \GraphInit
            \Vertex[x=-1, y=1, Lpos=180]{1}
            \Vertex[x=-1, y=0, Lpos=180]{2}
            \Vertex[x=-1, y=-1, Lpos=180]{3}
            \Vertex[x=0, y=0, Lpos=-90]{4}
            \Vertex[x=1, y=0, Lpos=-90]{5}
            \Vertex[x=2, y=1]{6}
            \Vertex[x=2, y=0]{7}
            \Vertex[x=2, y=-1]{8}
            \Edges(1,4,5,6)
            \Edges(2,4,3)
            \Edges(7,5,8)
            \node at (0.5,0.5){$\Gamma$};
        \end{tikzpicture} \quad $\overset{g}{\to}$ \quad
        \begin{tikzpicture}
            \GraphInit
            \Vertex[x=-1, y=1, Lpos=180]{7}
            \Vertex[x=-1, y=0, Lpos=180]{8}
            \Vertex[x=-1, y=-1, Lpos=180]{6}
            \Vertex[x=0, y=0, Lpos=-90]{5}
            \Vertex[x=1, y=0, Lpos=-90]{4}
            \Vertex[x=2, y=1]{1}
            \Vertex[x=2, y=0]{3}
            \Vertex[x=2, y=-1]{2}
            \Edges(1,4,5,6)
            \Edges(2,4,3)
            \Edges(7,5,8)
            \node at (0.5,0.5){$\Gamma$};
        \end{tikzpicture}
    \end{center}
\end{slide}

\section{Concluding remarks}

% \subsection{Membership testing}

% \begin{slide}
%     Test
% \end{slide}

% \subsection{References}

% \begin{slide}
%     \begin{itemize}
%         \item Dirichlet convolution: \url{https://www.johndcook.com/blog/2021/10/24/dirichlet-mobius/}
%         \item Dirichlet convolution: \url{https://en.wikipedia.org/wiki/Dirichlet_convolution}
%         \item M\"obius function: \url{https://en.wikipedia.org/wiki/M\%C3\%B6bius_function}
%         \item M\"obius inversion revisited (USC): \url{https://dornsife.usc.edu/assets/sites/1176/docs/PDF/430F19/430_Lecture_25.pdf}
%     \end{itemize}
% \end{slide}

% \subsection{References}

% \begin{slide}
%     \nocite{*}
%     \renewcommand{\bibname}{References}
%     \bibliographystyle{plain} % BibTeX only
%     \bibliography{references} % BibTeX only
% \end{slide}

\end{document}