\subsection{Representing the cube and its operations}

\begin{slide}
    Rubik's cube has 6 faces, each with $3 \times 3$ small \textit{stickers}.

    \onslide<2->{In \textbf{solved state} $1$, label stickers (except each centre) using $[48]$:}

    \begin{center}
        \includegraphics<1|handout:0>{graphics/rubiks_cube_net_empty.tikz}%
        \includegraphics<2->{graphics/rubiks_cube_net.tikz}%
    \end{center}

    \onslide<3->{6 \textbf{generators} (\textit{moves} in CC): $U,L,F,R,B,D$ (rot. \textit{clockwise}).}
\end{slide}

\begin{slide}
    From \textit{solved state} $1$, consider $F$ which rotates front face clockwise:

    \begin{center}
        \includegraphics<1|handout:0>{graphics/rubiks_cube_net.tikz}%
        \includegraphics<2->{graphics/rubiks_cube_net_front.tikz}%
    \end{center}

    \vspace{-1cm}
    \onslide<2->{\begin{multline*}
            F = (17,19,24,22)(18,21,23,20)( 6,25,43,16)\\
            ( 7,28,42,13)( 8,30,41,11) \in \Sym(48).
        \end{multline*}}
\end{slide}

\subsection{The Rubik's group of permutations}

\begin{slide}
    Generators as permutations of labels $[48]$:

    {\scriptsize
    \begin{itemize}
        \item $U = ( 1, 3, 8, 6)( 2, 5, 7, 4)( 9,33,25,17)(10,34,26,18)(11,35,27,19)$
        \item $L = ( 9,11,16,14)(10,13,15,12)( 1,17,41,40)( 4,20,44,37)( 6,22,46,35)$
        \item $F = (17,19,24,22)(18,21,23,20)( 6,25,43,16)( 7,28,42,13)( 8,30,41,11)$
        \item $R = (25,27,32,30)(26,29,31,28)( 3,38,43,19)( 5,36,45,21)( 8,33,48,24)$
        \item $B = (33,35,40,38)(34,37,39,36)( 3, 9,46,32)( 2,12,47,29)( 1,14,48,27)$
        \item $D = (41,43,48,46)(42,45,47,44)(14,22,30,38)(15,23,31,39)(16,24,32,40)$
    \end{itemize}} \pause

    \textbf{Operation} is sequence of generators and inverses. E.g. $RUR^{-1}U^{-1}$, \pause $URU^{-1}L^{-1}UR^{-1}U^{-1}L$, \pause $RUR^{-1}URU^2R^{-1}U^2$, \pause $1 = ()$.

    \begin{definition}[Rubik's group]
        \vspace{0pt}
        $\RC = \langle U,L,F,R,B,D \rangle \leq \Sym(48)$ is permutation group of degree 48, called \textbf{Rubik's group}.
    \end{definition}

    Clearly $\RC$ is finite, but what is $|\RC|$?
\end{slide}

% \begin{slide}
%     \textit{In cubing community:} operations called \textit{move sequences}. Inverse generators (also \textit{``moves''} in CC) written $U',L',F',R',B',D'$ (instead of $U^{-1}$ etc.); powers written $U2,R2$ etc. (instead of $U^2,R^2$). \pause

%     \textit{Recall:} $\sigma = \tau$ in $\Sym(n)$ iff $i^\sigma = i^\tau$ for all $i \in [n]$. \pause

%     Operations \textit{don't generally commute}: $RU \neq UR$ since

%         {\scriptsize
%             \begin{itemize}
%                 \item $R = (25,27,32,30)(26,29,31,28)( 3,38,43,19)( 5,36,45,21)( 8,33,48,24)$
%                 \item $U = ( 1, 3, 8, 6)( 2, 5, 7, 4)( 9,33,25,17)(10,34,26,18)(11,35,27,19)$
%             \end{itemize}}

%     \vspace{-1cm}
%     $$19^{RU} = \pause (19^R)^U = \pause 3^U = \pause 8 \quad\text{but}\quad 19^{UR} = \pause (19^U)^R = \pause 11^R = \pause 11.$$
% \end{slide}

\begin{slide}
    \texttt{GAP} code to define generators and $\RC = \langle U,L,F,R,B,D \rangle$ (as \texttt{G}):

    {\footnotesize\lstinputlisting{code/rubiks_def.gap}} \pause

    \texttt{Order} cmd: $|\RC| = 43\,252\,003\,274\,489\,856\,000 \approx 4.3 \cdot 10^{19}$. \textit{How?}
\end{slide}

\subsection{Orbits in the Rubik's group}

\begin{slide}
    \begin{center}
        \scalebox{0.6}{\includegraphics{graphics/rubiks_cube_net.tikz}}
    \end{center}

    \onslide<1>
    \scriptsize\lstinputlisting{code/rubiks_orbit_stab.gap}

    \normalsize Two $\RC$-orbits: corner stickers $1^\RC$, edge stickers $2^\RC$.
\end{slide}

\subsection{Transitive action on corners}

\begin{slide}
    $\RC$ acts transitively on corner stickers $1^\RC$. In this action:

    \begin{center}
        \scalebox{0.6}{\includegraphics{graphics/rubiks_cube_net_corners.tikz}}
    \end{center}

    \vspace{-1cm}
    \begin{multline*}
        \Sigma = \{\overbrace{\{1,35,9\}}^{UBL},\overbrace{\{6,11,17\}}^{ULF},\overbrace{\{40,46,14\}}^{BDL},\overbrace{\{27,3,33\}}^{RUB},\\
        \underbrace{\{8,25,19\}}_{URF},\underbrace{\{16,41,22\}}_{LDF},\underbrace{\{32,48,38\}}_{RDB},\underbrace{\{24,43,30\}}_{FDR}\}
    \end{multline*}
    is block system for \textit{maximal} block $\{8,25,19\}$ ($URF$ corner).
\end{slide}

\begin{slide}
    \begin{center}
        \scalebox{0.6}{\includegraphics{graphics/rubiks_cube_net_corners.tikz}}
    \end{center}

    $\RC$ acts primitively on $\Sigma$ (degree 8); $g \in \RC$ induces perm of $\Sigma$, e.g.
    $$F \mapsto (\underbrace{FUL}_{FDL^F},\underbrace{FUR}_{FUL^F},\underbrace{FDR}_{FUR^F},\underbrace{FDL}_{FDR^F}) \in \Sym(\Sigma).$$
    Overall, $\RC$ induces every perm of $\Sigma$ (so $\Sym(8)$ ``is'' quotient of $\RC$).
\end{slide}