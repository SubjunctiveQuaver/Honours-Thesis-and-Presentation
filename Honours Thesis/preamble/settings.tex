% Default settings
\allowdisplaybreaks
\frenchspacing
% \renewcommand{\familydefault}{\sfdefault} % Sans Serif
\renewcommand{\arraystretch}{0.8}

% Settings to run at start of document 
\AtBeginDocument{
    \setlength{\doublerulesep}{0.5pt}
    \setlength{\textfloatsep}{-1cm}
    \setlength{\abovedisplayskip}{5pt}
    \setlength{\belowdisplayskip}{5pt}}

% Paragraph spacing settings
\linespread{1.2}
\setlength{\parskip}{0.5em}
% \setlength{\parindent}{0pt} % No paragraph indent

% Header and footer
\pagestyle{fancy}
\lhead{}
\rhead{}
\fancyhead[LE]{\itshape\nouppercase\leftmark}
\fancyhead[LO,RE]{}
\fancyhead[RO]{\itshape\nouppercase\rightmark}
% \lhead{\TITLE}
% \rhead{\DATE}
% \lfoot{\VERSION}
% \rfoot{\AUTHOR}
\patchcmd{\chapter}{\thispagestyle{plain}}{\thispagestyle{fancy}}{}{}

% Theorems
% \theoremstyle{definition}
% \theoremsymbol{\ensuremath\square}
\theorempostskip{\topskip}
% \theorempostskip{0pt}
\declaretheorem[name=Definition, refname=Definition, numberwithin=chapter, style=definition]{definition}
\declaretheorem[name=Axiom, refname=Axiom, numberwithin=section]{axiom}
\renewcommand{\theaxiom}{\arabic{axiom}}
\declaretheorem[name=Proposition, refname=Proposition, sibling=definition]{proposition}
\declaretheorem[name=Theorem, refname=Theorem, sibling=definition]{theorem}
\declaretheorem[name=Corollary, refname=Corollary, sibling=definition]{corollary}
\declaretheorem[name=Lemma, refname=Lemma, sibling=definition]{lemma}
\declaretheorem[name=Remark, refname=Remark, sibling=definition, style=remark]{remark}
\declaretheorem[name=Algorithm, refname=Algorithm, sibling=definition]{algorithm}
\declaretheorem[name=Conjecture, refname=Conjecture, sibling=definition]{conjecture}
\declaretheorem[name=Example, refname=Example, sibling=definition, style=definition]{example}


% Package settings

%% enumerate
% \setlist[itemize]{topsep=0pt}
% \setlist[enumerate]{topsep=0pt}
\setlist[itemize]{topsep=0pt, itemsep=0pt}
\setlist[enumerate]{topsep=0pt, itemsep=0pt}

%% hyperref
\hypersetup{colorlinks=true, linkcolor=blue}

%% tcolorbox
\tcbuselibrary{breakable}
\newtcolorbox{framedbox}{standard jigsaw, opacityback=0, boxrule=1pt, breakable, before skip=8pt, after skip=8pt}

%% tkz-graph
% \GraphInit[vstyle=Hasse]
\GraphInit[vstyle=Welsh]
\SetGraphUnit{1}
\SetVertexMath
\tikzset{VertexStyle/.append style={minimum size=8pt}}

%% listings
% Code block styles (listings)
\definecolor{codegreen}{rgb}{0,0.6,0}
\definecolor{codegray}{rgb}{0.5,0.5,0.5}
\definecolor{codepurple}{rgb}{0.58,0,0.82}
\definecolor{backcolour}{rgb}{0.95,0.95,0.92}
\lstdefinestyle{default}{
    basewidth={.5em,0.5em},
    basicstyle=\ttfamily,
    breakatwhitespace=false,
    breaklines=true,
    captionpos=b,
    xleftmargin=15pt,
    framexleftmargin=2pt,
    framexrightmargin=2pt,
    keepspaces=true,
    numbers=left,
    numbersep=5pt,
    tabsize=4,
    columns=fullflexible,
    backgroundcolor=\color{backcolour},
    commentstyle=\itshape\color{codegreen},
    keywordstyle=\color{magenta},
    numberstyle=\footnotesize\itshape\color{black}\emptyaccsupp,
    stringstyle=\color{codepurple}
}
\lstset{style=default}