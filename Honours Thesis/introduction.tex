\section{Assumed knowledge}

This \thesis{} will assume knowledge of content covered in the (former) undergraduate unit MTH3121 -- Algebra and number theory I, run at Monash University until 2021, in particular the group theory half of the unit. It covers basic group theoretic concepts, such as isomorphism, homomorphisms, subgroups, cosets, Lagrange's theorem (see appendix), normal subgroups, quotient groups, and the first isomorphism theorem (see appendix), and studies cyclic groups, dihedral groups, symmetric groups and alternating groups. It did not cover group actions, bases, and related topics, which are therefore included here in the following chapters.

\section{Motivation}

\added{1}{Traditionally, a major goal of research in algebra has been to answer, in various ways, the question of finding all the algebraic structures that satisfy particular axioms, as noted in \cite{cannon_havas1992}. One such major result in this area was the famous classification of finite simple groups, which are finite groups whose only normal subgroups are the trivial ones. In \cite{solomon2001}, it is noted that around the year 2000, computational methods for groups were useful in calculating and classifying finite groups of various orders. This has all stemmed from the development of computational methods in algebra, especially in group theory, since the early 1970s, to compute information about groups for various purposes. In \cite{cannon_havas1992}, Cannon and Havas note that the state of development of computational group theory is relatively advanced, and has seen applications not only in the study of groups, but also in many other branches of mathematics which use group theoretic methods, for example differential equations, graph theory, number theory and topology.

    Cannon and Havas observe in \cite{cannon_havas1992} that the study of permutation groups is one of the areas in computational group theory that has traditionally seen high activity. A permutation group is a subgroup of the symmetric group $\Sym(\Omega)$ on some set $\Omega \neq \emptyset$, which is the group of all bijections $\Omega \to \Omega$ under composition. Permutation groups are particularly nice to work with computationally, as their elements can be explicitly identified as permutations on a set, and computations in the group can be easily performed by composing these permutations, say by a computer. Various computational packages such as \texttt{GAP} have been developed, which have implemented permutation groups computationally. Solomon observes in \cite{solomon2001} that computational methods for permutation groups have been considered in the classification problem for finite simple groups, among other applications. Every finite group can be represented as a permutation group by Cayley's theorem, but the degree (size of the set being permuted) of a permutation group representation is often small.

    Let $G \leq \Sym(\Omega)$ be a permutation group. To study $G$, the notion of group actions is particularly useful, since elements of $G$ are permutations of $\Omega$, so $G$ acts naturally on $\Omega$ by permutation. This also allows us to explore the notion of bases and stabiliser chains, as introduced and implemeneted by Sims in \cite{sims1970}: a base $B$ is a list of elements of $\Omega$ such that the identity $1_{\Sym(\Omega)}$ is the only permutation that fixes each element in $B$, and a stabiliser chain is a subgroup series, with each group in the series being the stabiliser of a base element in the preceding group. \added{2}{(See section \ref{sec:bases_stabiliser_chains} for details and examples.)} These notions allow us to answer a few natural questions about permutation groups, in particular \added{2}{determining} the order of $G$, finding a generating set for $G$ when realised as a group of permutations of certain objects (for example, vertices of an $n$-gon and vertices in a graph), \added{2}{determining} membership of arbitrary permutations $g \in \Sym(\Omega)$ in the subgroup $G$, and generating random elements of $G$. These preliminary questions are illustrated and addressed in this \thesis{}.

    Bases and stabiliser chains have been \added{2}{an} instrumental tool in developing group theoretic algorithms, and effectively facilitiate the storing and analysing of large permutation groups. Given the notion of bases and stabiliser chains, it is natural to ask about the possible sizes of a base, and if there are bounds on the size, perhaps if there is an efficient way to find a minimum (smallest) base. For instance, it can be easily seen that for $G = \Sym(n)$ the symmetric group on $n$ elements $\Omega = \{1,\dotsc,n\}$, a minimum base has size $n-1$. However, for a cyclic group $G = C_n$ (that also acts naturally on $\Omega$), any subset of $\Omega$ with one element is a base for $G$. Since every element in a permutation group is completely determined by its action on a base, a small base can reduce the space required to store the group.

    In \cite{blaha1992}, Blaha \added{3}{examines} the question of whether a polynomial time greedy algorithm suggested in \cite{brown1989} always finds a minimum base for a permutation group $G$, to answer the question ``for what $r$ does a permutation group $G$ have a base of size at most $r$?'' He shows that this is not the case, and even when restricted to elementary abelian groups, the problem is NP-hard. \added{2}{However}, Blaha shows that the algorithm presented produces a base of size $O(\mathcal{M}(G)\log\log n)$ where $\mathcal{M}(G)$ is the size of a minimum base for $G \leq \Sym(n)$, and that in the worst case, this bound, \added{2}{which is strictly larger than $\mathcal{M}(G)$,} is actually attained.

    This \thesis{} largely builds up to a discussion of the results about the NP-hardness of the minimum base problem in \cite{blaha1992}. Firstly, a detailed introduction to group-theoretic concepts such as group actions, bases and stabiliser chains is presented. Then, necessary concepts from complexity theory are introduced. Next, a detailed discussion of \cite{blaha1992} expands on some of Blaha's proofs, up to the section on sharp bounds for bases. Future directions include completing the discussion of Blaha's paper \cite{blaha1992}, collecting various known results on minimum bases for other families of groups, and identifying special classes of examples where specific bounds or results can be newly found in relation to the NP-hardness of the minimum base problem and analysis of the greedy base algorithm.

    \added{3}{TODO: ADD MORE}}

% \begin{itemize}
%     \item What is the problem
%     \item Overview of CGT
%     \item Why are minimum bases desirable
%     \item Tradeoffs with size of transversals?
%     \item Blaha's paper
%     \item Recent advances; what is known
%     \item List notation (where?)
%     \item Emphasise that chapters 2 \& 3 are new?
% \end{itemize}