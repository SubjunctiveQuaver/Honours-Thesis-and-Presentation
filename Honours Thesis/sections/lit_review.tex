\section{Results for non-large base primitive permutation groups}

Suppose we have a finite primitive permutation group $G$ of degree $n$. What can we say about $|G|$? (This problem attracted a lot of attention in the 19th century, as noted in \cite{moscatiello_roney-dougal2021}.) Of course, $|G| \leq n!$, and in the case that $G = \Sym(n)$ which is primitive, we have equality. In the case that $G = \Alt(n)$ which is also primitive, we have $|G| = n!/2 = O(n!)$, which suggests that even in nontrivial cases, the bound cannot be significantly improved on without further restrictions.

From \autoref{lem:blaha_nonredundant_size}, we have that $|G| \leq n^{b(G)}$. Thus, we may find upper bounds on $|G|$ by finding upper bounds on $b(G)$. One of the first results in this direction came in 1889, when Bochert proved in \cite{bochert1889} that for a primitive group $G$ of degree $n$ not containing the alternating group $\Alt(n)$, then $b(G) \leq n/2$. Compare this to \autoref{eg:symmetric_group_base} and \autoref{eg:alternating_group_base}, where we showed that $b(\Sym(n)) = n-1$ and $b(\Alt(n)) = n-2$, so the condition of not containing the alternating group immediately leads the upper bound on $b(G)$ improving by a factor of 2. Of course, since $\Alt(n)$ has index 2 in $\Sym(n)$, it is maximal, and thus every primitive group $G$ of degree $n$, apart from $\Alt(n)$ and $\Sym(n)$, has $b(G) \leq n/2$.

Improvements to these bounds have been made for primitive groups with certain additional properties. In 1984, Liebeck used the \hyperref[thm:cfsg]{classification of finite simple groups} and the \hyperref[thm:onan-scott]{O'Nan-Scott theorem} to improve the bound on $b(G)$ for \textit{non-large base} permutation groups.

\begin{definition}\label{def:large_base}
    A permutation group $G$ of degree $n$ is \textbf{large base} if there are integers $m$ and $r \geq 1$ with
    $$\Alt(m)^r \unlhd G \leq \Sym(m) \wr \Sym(r),$$
    where $\Sym(m)$ \hyperref[eg:product_action_Sm_subsets]{acts on $k$-element subsets} of $\{1,\dotsc,m\}$ for some $k$, and the \hyperref[def:wreath_product]{wreath product} has the \hyperref[def:product_action]{product action} of degree $n = \binom{m}{k}^r$ if $r > 1$.

    (Note that here we mean that $G$ contains a subgroup that is isomorphic to $\Alt(m)^r$, and $G$ is isomorphic to a subgroup of the wreath product $\Sym(m) \wr \Sym(r)$. Such convention is common in the literature, where inclusion is up to isomorphism.)
\end{definition}

By this definition, $\Alt(n)$ and $\Sym(n)$ with their natural actions are large base, where we choose $m = n$ and $r = 1$; recall that $\Sym(n) \wr \Sym(1) \cong \Sym(n)$, and $\Alt(n) \unlhd \Sym(n)$. Thus, the non-large base primitive groups exclude $\Alt(n)$ and $\Sym(n)$, and there is no conflict with Bochert's result in \cite{bochert1889}. The following result by Liebeck is found in \cite{liebeck1984}.

\begin{theorem}[Liebeck, 1984]\label{thm:liebeck1984}
    Let $G$ be a primitive group of degree $n$. Then one of the following holds:
    \begin{enumerate}[(i)]
        \item $G$ is \hyperref[def:large_base]{large base}; or
        \item $b(G) < 9\log n$.
    \end{enumerate}
\end{theorem}

The result was proven in context improving lower bounds on $\mu(G)$, the \textbf{minimal degree} of $G \leq \Sym(\Omega)$, which is the smallest number of points moved by any non-identity element in $G$, i.e.
$$\mu(G) = \min_{g \in G \setminus \{1\}}|\{\alpha \in \Omega : \alpha^g \neq \alpha\}|.$$
For example, $\mu(\Alt(n)) = 3$ for $n \geq 3$, since $\Alt(n)$ contains 3-cycles (which move 3 points) but no transpositions (which move 2 points). From the relation that $b(G)\mu(G) \geq n$ for any transitive group $G$ of degree $n$ (see \cite{cameron1984} for proof), Liebeck concludes that either $G$ is large base, or that $\mu(G) > n/(9 \log n)$. The best result that was previously available, due to Babai in \cite{babai1981}, was that $\mu(G) > (1/2)(\sqrt{n} - 1)$ as long as $\Alt(n) \not\leq G$ (which would give, by properties of \hyperref[def:big_O_notation]{big-O},
$$b(G) < 2\sqrt{n}\left(1 - \frac{1}{\sqrt{n}}\right)^{-1} = 2\sqrt{n}\left(1 + O\left(\frac{1}{\sqrt{n}}\right)\right) = 2\sqrt{n} + O(1) = O(\sqrt{n}),$$
i.e. a square-root bound on $b(G)$); the logarithmic bound on $b(G)$ in Liebeck represented a major improvement, and has been seen as a remarkable result \cite{moscatiello_roney-dougal2021}.

Liebeck's proof of \autoref{thm:liebeck1984} in \cite{liebeck1984} is approached in four steps. Firstly, a reduction to simple socle is performed: supposing the theorem is true for primitive groups with a nonabelian simple socle, then considering arbitrary primitive $G$ to prove part (i) of the theorem. Then, by considering cases in the O'Nan-Scott theorem, Liebeck shows that it suffices to assume $\Soc(G) = T$ is simple. The second step uses the \hyperref[thm:cfsg]{classification of finite simple groups} to analyse the structure of the nonabelian simple socle $T$, leading to the conclusion that $T$ is an alternating group, a group of Lie type, or a sporadic group, or that $|G| < n^9$. The third step shows that these non-large base groups from step two satisfy (ii) in \autoref{thm:liebeck1984}, and step four uses the previous steps to complete the proof.

In 2019, Halasi, Liebeck and Mar\'oti show in Theorem 1.1 of \cite{halasi2019} that the minimal base size of an arbitrary primitive group of degree $n$ satisfies $b(G) \leq 2(\log|G|/\log n) + 24$, with the multiplicative constant 2 being best possible, answering the well-known \textit{Pyber's conjecture} of the existence of a universal constant $c$ with $b(G) < c(\log|G|/\log n)$ for all primitive groups $G$, which was open until initially proven in 2018. In particular, \cite{halasi2019} shows that for most non-large base primitive groups $G$, the minimal base size satisfies $b(G) \leq 2\lfloor\log n\rfloor + 26$; in 2020, ``most'' was improved to \textit{``all''} \cite{moscatiello_roney-dougal2021}.

In 2021, building on the previously mentioned results of Liebeck in \cite{liebeck1984} and Halasi, Liebeck and Mar\'oti in \cite{halasi2019}, Moscatiello and Roney-Dougal improve the bound on $b(G)$ to $\lceil\log n\rceil + 1$ for non-large base primitive groups, with one exception. The following is Theorem 1 in this paper \cite{moscatiello_roney-dougal2021}; it is the main result.

\begin{theorem}[Moscatiello and Roney-Dougal, 2021]\label{thm:moscatiello_roney-dougal2021_1}
    Let $G$ be primitive group of degree $n$. If $G$ is non-\hyperref[def:large_base]{large base}, then either:
    \begin{enumerate}[(i)]
        \item $G$ is the Mathieu group $M_{24}$ in its 5-transitive action of degree 24; or
        \item $b(G) \leq \lceil\log n\rceil + 1$.
    \end{enumerate}
    Moreover, there are infinitely many $G$ that are non-large base, for which $b(G) > \log n + 1$.
\end{theorem}

The \textbf{Mathieu groups} are five sporadic simple permutation groups $M_{11}$, $M_{12}$, $M_{22}$, $M_{23}$ and $M_{24}$ of degree 11, 12, 22, 23 and 24 respectively, with multiply transitive actions. They were introduced by Mathieu between 1861 and 1873, and were the first sporadic simple groups to be discovered. The final observation in \autoref{thm:moscatiello_roney-dougal2021_1} considers groups $G$ with $\log n + 1 < b(G) \leq \lceil\log n\rceil + 1$ (aside from the one exception of $M_{24}$), which would require that $b(G) = \lceil\log n\rceil + 1$ and that $n$ is not a power of 2.

Next, we present Theorem 5 from \cite{moscatiello_roney-dougal2021}, which is a more detailed version of \autoref{thm:moscatiello_roney-dougal2021_1}. It mostly investigates the case in (ii) above of $\log n + 1 \leq b(G) \leq \lceil\log n\rceil + 1$, which collapses to $b(G) = \lceil\log n\rceil + 1$.

\begin{theorem}[Moscatiello and Roney-Dougal, 2021]\label{thm:moscatiello_roney-dougal2021_5}
    Let $G$ be primitive subgroup of $\Sym(\Omega)$ with $|\Omega| = n$. Then $b(G) \geq \log n + 1$ if and only if $G$ is one of following:
    \begin{enumerate}[(i)]
        \item A subgroup of $\AGL_d(2)$ with $b(G) = d + 1 = \log n + 1$.
        \item The \hyperref[def:symplectic_group]{symplectic group} $\Sp_d(2)$ with $d \geq 4$; then $\log n + 1 < b(G) = \lceil\log n\rceil + 1$. % (acting on cosets of $\GO^-_d(2)$)
        \item A Mathieu group $M_n$ in its natural permutation representation with $n \in \{12,23,24\}$. If $n = 12$ or $23$ then $b(G) = \lceil\log n\rceil + 1$, while if $n = 24$ then $b(G) = 7 > \lceil\log n\rceil + 1$.
    \end{enumerate}
\end{theorem}

The infinite number of primitive groups $G$ of degree $n$ with $b(G) > \log n + 1$, as mentioned in \autoref{thm:moscatiello_roney-dougal2021_1}, is demonstrated by case (ii) in the more detailed theorem above; in case (i), it is conditional on the existence of such subgroups $G \leq \AGL_d(2)$ with $b(G) = d + 1$. Thus, the authors conclude the paper \cite{moscatiello_roney-dougal2021} with a question:

\begin{question}[Moscatiello and Roney-Dougal, 2021]\label{qn:moscatiello_roney-dougal2021}
    Which primitive groups $G \leq \Sym(n)$ satisfy $b(G) = \log n + 1$?
\end{question}

Of course, we require that $\log n$ is an integer, so that $n$ is a power of 2. By \autoref{thm:moscatiello_roney-dougal2021_5}, we must have $G \leq \AGL_d(2)$ for some $d$ (indeed, $\AGL_d(2)$ acts on $n = 2^d$ points; all other cases in the theorem have $b(G) > \log n + 1$), and if $d$ is even then it is noted in \cite{moscatiello_roney-dougal2021} that groups such as the split extension $2^d : \Sp_d(2)$ have this property (we will assume this in the rest of this \thesis{}). Are there any more such groups? What about for odd $d$? Investigating this question forms the remainder of this \thesis{}.

\section{Primitive subgroups of affine groups with given minimal base size}

The main result of this section is as follows.

\begin{theorem}\label{thm:new_result}
    Let $G$ be a primitive subgroup of $\AGL_d(2)$ for some $d$, with the induced action of degree $n = 2^d$.
    \begin{enumerate}[(i)]
        \item For $d = 1$, there is no such $G$ with $b(G) = d + 1 = \log n + 1$.
        \item For odd $3 \leq d \leq 9$ and $d = 2$, if $b(G) = d + 1 = \log n + 1$, then $G$ is the affine group $\AGL_d(2)$.
        \item For even $4 \leq d \leq 10$, if $b(G) = d + 1 = \log n + 1$, then $G$ is either $\AGL_d(2)$ or $2^d : \Sp_d(2)$.
    \end{enumerate}
\end{theorem}

Our approach to this result is to use \texttt{GAP} to iterate over potential primitive groups $G \leq \AGL_d(2)$ for some $d$ and compute a bound on $b(G)$ to eliminate those groups for which a base of size less than $d + 1$ is found. Using the \texttt{FinInG} (finite incidence geometry) package, we use a permutation representation of $\AGL_d(2)$ (a subgroup of $\Sym(2^d)$). Note that if a base of size $r$ is found for $G$, then $b(G) \leq r$, so if $r < d + 1$, we may eliminate such $G$.

To compute a base for $G$, \texttt{GAP} has the built-in command \texttt{BaseOfGroup}. However, in various test cases, we find that when compared to our implementation of the greedy base algorithm (\autoref{alg:blaha_greedy_base}) analysed in Blaha \cite{blaha1992}, the greedy algorithm sometimes finds a smaller base than the built-in command. For instance, when running a modified version of the program that proves \autoref{thm:new_result} with $d = 6$, a permutation group $G$ of degree 64 and order $3\,612\,672$ (in fact, a primitive subgroup of $\AGL_6(2)$) was found, for which \texttt{BaseOfGroup} finds a base of size $7 = d + 1$, while \texttt{GreedyBase} finds a base of size $5 < d + 1$; this led to a number of false positives in an initial analysis. (See the \hyperref[app:greedy_better_than_default]{appendix} for \texttt{GAP} code verifying this.)

Thus, in the procedure to prove the main theorem, we use our implementation of \texttt{GreedyBase} instead of \texttt{BaseOfGroup} to compute a base for $G$ (see appendix for implementation). Both algorithms are fast; as discussed earlier, the greedy base algorithm is a polynomial-time algorithm. Of course, neither algorithm necessarily finds a minimum base for $G$, but in conjunction with \autoref{thm:blaha_greedy_size}, we see that a greedy base for $G$ has size at most $\lceil b(G)\log\log n \rceil + b(G) = \lceil b(G)\log d \rceil + b(G) = O(b(G)\log d)$, which is fine for our purposes (and in practice does not lead to any issues for our result --- all subgroups with $b(G) < d + 1$ have been identified).

The remainder of this section is dedicated to our approach to proving \autoref{thm:new_result}, which is an original proof. Note that all subgroups take the induced subgroup action.

\begin{proof}[Proof of main theorem]
    First, we consider $d = 1$ in case (i). First note that $|\AGL_1(2)| = 2^1(2^1 - 1) = 2$, so if $G$ is a primitive subgroup then $G = \AGL_1(2)$ (the trivial subgroup is not transitive, thus not primitive). By \autoref{prop:agl_as_subgrp_of_gl}, $b(\AGL_1(2)) = 1 < d + 1$. This proves the result for case (i).

    Fix $2 \leq d \leq 10$; first, note that if $G$ is the affine group $\AGL_d(2)$, then by \autoref{prop:agl_as_subgrp_of_gl} $b(G) = d + 1$. It remains to check proper primitive subgroups $G$ of $\AGL_d(2)$ to identify those with $b(G) = d + 1$. To do so, we adopt a ``brute force approach'' with a number of optimisations. It uses the following observations to avoid checking \textit{all} primitive subgroups of $\AGL_d(2)$, especially at any given stage in the process:

    \begin{enumerate}
        \item \textit{Once we find a subgroup $G$ with $b(G) < d + 1$, we do not need to consider any of its (proper) subgroups.} This is due to \autoref{lem:base_of_subgroup}, which says that if $H \leq G$, then $b(H) \leq b(G) < d + 1$.
        \item \textit{If we find a non-primitive (imprimitive or intransitive) subgroup $G$, we do not need to consider any of its subgroups.} This is because subgroups of non-primitive groups (that act on $\Omega$) are non-primitive: suppose $\Sigma = \{\Delta^g : g \in G\}$ is a system of imprimitivity for $G$ transitive, where $\Delta$ is a block. Then for transitive $H \leq G$, $\Delta$ is clearly a block for the induced $H$-action, and $\Sigma = \{\Delta^h : h \in H\}$ is a system of imprimitivity for $H$. If $G$ is intransitive, then there is $\alpha \in \Omega$ with $\alpha^G \neq \Omega$; then $\alpha^H \subseteq \alpha^G$ for $H \leq G$, so $H$ is intransitive.
        \item \textit{If we have $G$ with $b(G) = d + 1$, we may recursively check maximal subgroups.} This is because $\AGL_d(2)$ is finite, so for arbitrary $H \leq G$, we have a \textit{finite} subgroup series
              $$H = H_0 < H_1 < \dotsb < H_k = G$$
              where each $H_i$ is maximal in $H_{i + 1}$ (else insert intermediate groups in the series). Thus, if $b(H) = d + 1$, then we must have $b(H_i) = d + 1$ for all $i$, by observation 1 above; recursively checking maximal subgroups will guarantee that we eventually find $H$.
        \item \textit{At each stage, it suffices to check maximal subgroups up to conjugacy.} This is because \autoref{prop:conjugate_subgroups_bases} implies that $b(G^\sigma) = b(G)$ for all $\sigma \in \Sym(\Omega)$, where $G$ acts on $\Omega$, so that we may consider a representative from each conjugacy class of subgroups. Moreover, if $G$ is maximal, then $G^\sigma$ is also maximal by \autoref{lem:conjugate_of_maximal_subgroup_is_maximal}, which implies that checking only maximal subgroups up to conjugacy will not overlook any conjugacy classes of subgroups. This observation also leads to a significant improvement over the na\"ive approach, since the size of a conjugacy class of subgroups may be large. However, it means that the subgroups found by the process are unique only up to conjugacy.
    \end{enumerate}

    These observations can be adapted for any group, to find (primitive) subgroups $H$ with $b(H) > r$ for any given $r$, by simply replacing ``$d$'' with ``$r$''. If we wish to ignore the condition of primitivity, we may simply ignore observation 2 above.

    Below, we present the generalised recursive algorithm for an arbitrary permutation group $G$, based on the above observations, that allows us to find all (primitive) proper subgroups $H \leq G$ with $b(H) > r$. In particular, if $b(H) > r$, then the algorithm identifies $H$ up to conjugacy; however, it could also incorrectly identify $H$ with $b(H) \leq r$ in the process. Any (conjugacy classes of) subgroups $H$ omitted certainly have $b(H) \leq r$. (An alternative is to use a brute force approach to compute $b(H)$ exactly, but this is much more inefficient, and comes with few, if any, advantages.)

    \begin{algorithmic}[1]
        \Procedure{GetSubgrpBase}{$G,r,L$}\Comment{Finds all primitive $H \leq G$ with $b(H) > r$, storing in list $L$}
        \State $\tilde L \gets [\ ]$\Comment{Initialise new list for newly found candidates}
        \For{representatives $M < G$ of conjugacy classes of maximal subgroups of $G$}\Comment{Observation 4 above}
        \If{$M$ is primitive} $\ell \gets$ \textsc{Length}(\textsc{GreedyBase}($M$))\Comment{Can drop condition of primitivity here}
        \If{$\ell > r$} add $M$ to $L$ and $\tilde L$\Comment{Primitive candidate for $b(M) > r$}
        \EndIf
        \EndIf
        \EndFor
        \For{$H$ in $\tilde L$}
        \State $L \gets$ \textsc{GetSubgrpBase}$(H,r,L)$\Comment{Recursively run on candidates $H$ in $\tilde L$}
        \EndFor
        \State \Return $L$\Comment{$L$ now contains all $H < G$ with $b(H) > r$ up to conjugacy}
        \EndProcedure
    \end{algorithmic}

    Note that the algorithm does not check the initial group $G$ itself, so this must be performed separately. See the \hyperref[app:subgrps_base_len]{appendix} for an implementation in \texttt{GAP} using the functions \texttt{GetSubgrpBase} and \texttt{GetSubgrpAGLBase} (for the special case of $\AGL_d(2)$ and testing for $G \leq \AGL_d(2)$ with $b(G) > d$, i.e. $b(G) = d + 1$); these also print out various information about the subgroups identified in the process.

    Let us now return to the notation of \autoref{thm:new_result}, where $G$ is a primitive proper subgroup of $\AGL_d(2)$ for given $2 \leq d \leq 10$. Running \texttt{GetSubgrpAGLBase( d )} yields the following (see \hyperref[app:agl_output]{appendix} for output):

    \begin{itemize}
        \item If $d = 2$, no groups $G$ are found. (2 maximal primitive subgroups $M$ are considered, but each has $b(M) < d + 1$.)
        \item If $d = 3$, no groups $G$ are found. (3 maximal primitive subgroups $M$ are considered, but each has $b(M) < d + 1$.)
        \item If $d = 4$, one group $G$ is found, which must be $2^4 : \Sp_4(2)$.
        \item If $d = 5$, no groups $G$ are found. (2 maximal primitive subgroups $M$ are considered, but each has $b(M) < d + 1$.)
        \item If $d = 6$, one group $G$ is found, which must be $2^6 : \Sp_6(2)$.
        \item If $d = 7$, no groups $G$ are found. (2 maximal primitive subgroups $M$ are considered, but each has $b(M) < d + 1$.)
        \item If $d = 8$, one group $G$ is found, which must be $2^8 : \Sp_8(2)$.
        \item If $d = 9$, no groups $G$ are found. (5 maximal primitive subgroups $M$ are considered, but each has $b(M) < d + 1$.)
        \item If $d = 10$, one group $G$ is found, which must be $2^{10} : \Sp_{10}(2)$.
    \end{itemize}

    (Note that in each case for which a group $G$ is found for some $d$, it has the correct order to be $2^d : \Sp_d(2)$; to verify this, observe that its order is $2^d |\Sp_d(2)| = 2^d |\GO_{d+1}(2)|$ by \autoref{thm:symplectic_orthogonal_isom}, which we verify in \texttt{GAP}.)

    In the case that $d = 2$ or $3 \leq d \leq 9$ is odd, we see that there are no proper subgroups $G \leq \AGL_d(2)$ with $b(G) = d + 1$, which completes case (ii) of the theorem. In the remaining case that $4 \leq d \leq 10$ is even, we see that if $G \leq \AGL_d(2)$ satisfies $b(G) = d + 1$, then $G$ must be $2^d : \Sp_d(2)$; combining this with the observation by Moscatiello and Roney-Dougal in \cite{moscatiello_roney-dougal2021} that $b(2^d : \Sp_d(2)) = d + 1$ completes case (iii) and the entire proof.
\end{proof}

This result seems to suggest that the groups identified by Moscatiello and Roney-Dougal in \cite{moscatiello_roney-dougal2021} are possibly the only primitive groups with the given property; any counterexamples must be subgroups of the affine group over $\F_2$ of dimension at least 11, which are large groups, and finding such groups would perhaps be surprising. A limitation of our proof approach is that it only works practically for $d \leq 10$, due to memory and computational time constraints, but is nevertheless an interesting approach. In particular, the group identified for $d = 10$ has order $25\,410\,822\,678\,459\,187\,200 \approx 2.5 \cdot 10^{19}$ and thus has many subgroups. Without a more systematic way of describing its maximal primitive subgroups, this approach cannot be used for larger values of $d$: the case that $d = 9$ took 5 mins, the case that $d = 10$ took 100 mins, and the case that $d = 11$ took over 48 hours without returning a result.

A more efficient identification of maximal primitive subgroups in $\AGL_d(2)$, such as if a classification result exists, could lead to a vast improvement in the running time of the \texttt{GetSubgrpAGLBase} algorithm, and potentially lead to this approach being useful to verify results for larger values of $d$. This is because at each stage, the procedure often identified only 2--5 primitive subgroups; the majority of the time spent is \texttt{GAP} attempting to find maximal subgroups up to conjugacy. Additionally, other algorithms to find bases and minimal base size could be employed, but the greedy algorithm is already quite fast and performs well.

Based on this discussion and the behaviour in \autoref{thm:new_result}, we summarise and conclude with the following conjecture in the direction of \autoref{qn:moscatiello_roney-dougal2021}.

\begin{conjecture}\label{conj:new_result}
    A primitive group $G \leq \Sym(n)$ satisfies $b(G) = \log n + 1$ if and only if $G$ is one of the following:
    \begin{enumerate}[(i)]
        \item $n = 2^d$ with $d \geq 2$, and $G$ is the affine group $\AGL_d(2)$; or
        \item $n = 2^d$ with $d \geq 4$ even, and $G$ is the split extension $2^d : \Sp_d(2)$.
    \end{enumerate}
\end{conjecture}

Any counterexamples must be permutation groups of degree $2^d$ with $d \geq 11$ that are a subgroup of $\AGL_d(2)$.