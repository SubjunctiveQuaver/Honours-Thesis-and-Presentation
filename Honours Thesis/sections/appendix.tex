\chapter{Appendix --- theory of permutation groups}%\chaptermark{Appendix}

\textit{Note that Adobe Acrobat allows you to copy the following code without line numbers.} All code in this \thesis{} is implemented and tested in \texttt{GAP} version 4.12.0.

\section{Rubik's group}

Here is \texttt{GAP} code relevant to \autoref{eg:rubiks_group}. First we define the Rubik's group $G \leq \Sym(48)$.\label{app:rubiks_group}

\lstinputlisting{gap_code/rubiks_def.gap}

Then, we compute orbits, restrict the action to \texttt{GC} (which acts on corners), look at maximal blocks under this action (which are the corners), then analyse the primitive action on corners (which is \texttt{GCBhom}).

\lstinputlisting{txt_files/rubiks_group.gap}

The following code is relevant to \autoref{eg:rubiks_group_random}.

\lstinputlisting{txt_files/rubiks_group_membership.gap}

\section{Random elements and the constructive membership problem}

Here is an implementation of \autoref{alg:transversal_random_element} in \texttt{GAP} as the function \texttt{RandomElt}:\label{app:transversal_random_element}

\lstinputlisting{gap_code/random_elt.gap}

Here is an implementation of \autoref{alg:transversal_membership_test} in \texttt{GAP} as the function \texttt{Membership} (using two subfunctions):\label{app:transversal_membership_test}

\lstinputlisting{gap_code/membership.gap}

Here is the modified version of \texttt{RandomElt} which prints out intermediate choices of $t_i \in T_i$ and calculations $g \in G$, as used in \autoref{eg:automorphism_group_graph_random}:

\lstinputlisting{gap_code/random_elt_print.gap}

Here is the modified version of \texttt{Membership} which prints out values of $t_i \in T_i$ and $h$, as used in \autoref{eg:automorphism_group_graph_membership}:

\lstinputlisting{gap_code/membership_print.gap}

\subsection{Stabiliser chain for automorphism group of graph}

Here is some more output relevant to \autoref{eg:automorphism_group_graph_bsgs}:\label{app:automorphism_group_graph_bsgs}

\lstinputlisting{txt_files/automorphism_group_graph_bsgs_gap_appendix.gap}

Lines 28--41 describe all elements of $G = \Aut(\Gamma)$. Lines 62--67 describe all elements of $G_1 = \Stab_G(4)$.

\section{Wreath product and dihedral group}

Here is \texttt{GAP} code relevant to \autoref{eg:S2_wr_S2_isom_D8}:

\lstinputlisting{gap_code/d8_wreath_product.gap}

The following is the output which is summarised in the table in the example:

\lstinputlisting{txt_files/d8_wreath_product_gap_output.gap}

\chapter{Appendix --- computing bases of subgroups of affine groups}%\chaptermark{Appendix}

\section{Greedy base algorithm}

Below is \texttt{GAP} code for an implementation of the greedy base algorithm from \autoref{alg:blaha_greedy_base} as \texttt{GreedyBase}:\label{app:greedy_base}

\lstinputlisting{gap_code/greedy_base.gap}

The following example of a permutation group $G$ of degree 64 shows that \texttt{GreedyBase} sometimes finds a smaller base than the built-in command \texttt{BaseOfGroup}:\label{app:greedy_better_than_default}

\lstinputlisting{txt_files/greedy_better_than_default.gap}

\section{Program to find primitive subgroups with given minimal base size}

Below is \texttt{GAP} code for the recursive program that was used to identify conjugacy classes of (primitive) proper subgroups $G$ of $\AGL_d(2)$ (for given $d$) which may satisfy $b(G) = d + 1$. It includes the greedy base algorithm as described above (so that the code is fully self-contained).

Given $d \geq 1$, running \texttt{GetSubgrpAGLBase( d )} gives a list of candidate primitive subgroups of $\AGL_d(2)$ with minimal base size $d + 1$, according to the greedy base algorithm (which seemed to perform better than the built-in \texttt{BaseOfGroup} function which presented false positives).

The \texttt{GetSubgrpAGLBase} function relies on the function \texttt{GetSubgrpBase}; note that \texttt{GetSubgrpBase( r, G )} returns information about proper (primitive) subgroups $H$ of a group $G$ such that the greedy base algorithm finds no base of size at most $r$ (thus are candidates for $b(H) > r$). Thus, the program can be easily adjusted to answer similar questions for different families of groups.\label{app:subgrps_base_len}

\lstinputlisting{gap_code/subgrps_base_len.gap}

% Below is output for $d = 1$:

% \lstinputlisting{txt_files/agl_base_gap_output/1.gap}

\subsection{Program output for low dimensional affine groups}\label{app:agl_output}

Below is output for $d = 2$:

\lstinputlisting{txt_files/agl_base_gap_output/2.gap}

Below is output for $d = 3$:

\lstinputlisting{txt_files/agl_base_gap_output/3.gap}

Below is output for $d = 4$:

\lstinputlisting{txt_files/agl_base_gap_output/4.gap}

Below is output for $d = 5$:

\lstinputlisting{txt_files/agl_base_gap_output/5.gap}

Below is output for $d = 6$:

\lstinputlisting{txt_files/agl_base_gap_output/6.gap}

Below is output for $d = 7$:

\lstinputlisting{txt_files/agl_base_gap_output/7.gap}

Below is output for $d = 8$:

\lstinputlisting{txt_files/agl_base_gap_output/8.gap}

Below is output for $d = 9$:

\lstinputlisting{txt_files/agl_base_gap_output/9.gap}

Below is output for $d = 10$:

\lstinputlisting{txt_files/agl_base_gap_output/10.gap}

For even $d$, we may check that order is correct for $2^d : \Sp_d(2)$, by verifying that the groups identified have order equal to \texttt{2\^{}d * Order( GeneralOrthogonalGroup( d + 1, 2 ) )} using the isomorphism $\Sp_d(2) \cong \GO_{d+1}(2)$; this is the case in every scenario.

% \lstinputlisting{gap_code/agl_matrices.gap}