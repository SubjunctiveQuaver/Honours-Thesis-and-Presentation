A \textit{permutation group} $G$ of degree $n$ is a subgroup of the symmetric group $\Sym(\Omega)$ that acts on a set $\Omega$ of size $n$, where $\Sym(\Omega)$ is the set of permutations of $\Omega$. An ordered list $B = [\beta_1,\dotsc,\beta_r]$ of distinct elements of $\Omega$ is a \textit{base} for $G$ if the pointwise stabiliser $G_{(B)}$ is trivial; the minimal size of a base for $G$ is denoted $b(G)$. Every base has an associated \textit{stabiliser chain} $G = G^0 \geq G^1 \geq \dotsb \geq G^r = 1$, where each subgroup $G^i = G^{i-1}_{\beta_i}$ in the chain is a stabiliser of the previous subgroup; the concept of a \textit{BSGS} allows us to easily perform various computations with permutation groups, such as membership testing and random element generation. In this \thesis{}, we first analyse group actions and bases, and give an introduction to complexity theory. Then, we follow Blaha's proof in \cite{blaha1992} that the \textit{minimum base problem} is NP-hard even if restricted to cyclic groups or elementary abelian group, and discuss a greedy algorithm for computing a small base for $G$, which is not optimal but still useful. Subsequently, we consider various classification results in permutation group theory, such as the \textit{classification of finite simple groups} and the \textit{O'Nan-Scott theorem} for primitive groups, which we use to discuss recent progress in bounding the mminimal base size $b(G)$ for groups that are not \textit{large base}: a permutation group $G$ is large base if contains a power of an alternating group, and is itself embedded in a \textit{wreath product} of certain symmetric groups. Finally, we investigate a question from Moscatiello and Roney-Dougal in \cite{moscatiello_roney-dougal2021}, and using computational methods in \texttt{GAP} and the greedy base algorithm, we show that there are very few primitive subgroups $G \leq \Sym(2^d)$, where $1 \leq d \leq 10$, such that $b(G) = d + 1$. These groups are subgroups of the \textit{affine group} $\AGL_d(2)$, and there are none for $d = 1$, one for $d = 2$ and odd $3 \leq d \leq 9$, and two for even $4 \leq d \leq 10$, all up to conjugacy.