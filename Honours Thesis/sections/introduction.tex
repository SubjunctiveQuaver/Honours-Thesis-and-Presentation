\section{Assumed knowledge}

This \thesis{} will assume knowledge of content covered in the (former) undergraduate unit MTH3121 -- Algebra and number theory I, run at Monash University until 2021, in particular the group theory half of the unit. It covers basic group theoretic concepts, such as isomorphism, homomorphisms, subgroups, cosets, Lagrange's theorem, normal subgroups, quotient groups, and the first isomorphism theorem, and studies cyclic groups, dihedral groups, symmetric groups and alternating groups. It did not cover group actions, bases, and related topics, which are therefore included here in the following chapters.

\section{Motivation}

Traditionally, a major goal of research in algebra has been to answer, in various ways, the question of finding all the algebraic structures that satisfy particular axioms, as noted in \cite{cannon_havas1992}. One such major result in this area was the famous classification of finite simple groups, which are finite groups whose only normal subgroups are the trivial ones. In \cite{solomon2001}, it is noted that around the year 2000, computational methods for groups were useful in calculating and classifying finite groups of various orders. This has all stemmed from the development of computational methods in algebra, especially in group theory, since the early 1970s, to compute information about groups for various purposes. In \cite{cannon_havas1992}, Cannon and Havas note that the state of development of computational group theory is relatively advanced, and has seen applications not only in the study of groups, but also in many other branches of mathematics which use group theoretic methods, for example differential equations, graph theory, number theory and topology.

Cannon and Havas observe in \cite{cannon_havas1992} that the study of permutation groups is one of the areas in computational group theory that has traditionally seen high activity. A permutation group is a subgroup of the symmetric group $\Sym(\Omega)$ on some set $\Omega \neq \emptyset$, which is the group of all bijections $\Omega \to \Omega$ under composition. Permutation groups are particularly nice to work with computationally, as their elements can be explicitly identified as permutations on a set, and computations in the group can be easily performed by composing these permutations, say by a computer. Various computational packages such as \texttt{GAP} have been developed, which have implemented permutation groups computationally. Solomon observes in \cite{solomon2001} that computational methods for permutation groups have been considered in the classification problem for finite simple groups, among other applications. Every finite group can be represented as a permutation group by Cayley's theorem, and the degree (size of the set being permuted) of a permutation group representation is often small, \added{4}{especially compared to the size of the group.}

Let $G \leq \Sym(\Omega)$ be a permutation group. To study $G$, the notion of group actions is particularly useful, since elements of $G$ are permutations of $\Omega$, so $G$ acts naturally on $\Omega$ by permutation. Actions with only one orbit are said to be \textit{transitive}; if a transitive action has no nontrivial \textit{blocks}, then it is \textit{primitive}. This also allows us to explore the notion of bases and stabiliser chains: a base $B$ is a list of elements of $\Omega$ such that the identity $1_{\Sym(\Omega)}$ is the only permutation that fixes each element in $B$, and a stabiliser chain \added{4}{(as introduced and \added{4}{implemented} by Sims in \cite{sims1970})} is a subgroup series, with each group in the series being the stabiliser of a base element in the preceding group. \added{2}{(See section \ref{sec:bases_stabiliser_chains} for details and examples.)} These notions allow us to answer a few natural questions about permutation groups, in particular \added{2}{determining} the order of $G$, finding a generating set for $G$ when realised as a group of permutations of certain objects (for example, vertices of an $n$-gon and vertices in a graph), \added{2}{determining} membership of arbitrary permutations $g \in \Sym(\Omega)$ in the subgroup $G$, and generating random elements of $G$. These preliminary questions are illustrated and addressed in this \thesis{}.

Bases and stabiliser chains have been \added{2}{an} instrumental tool in developing group theoretic algorithms, and effectively facilitate the storing and analysing of large permutation groups. Given the notion of bases and stabiliser chains, it is natural to ask about the possible sizes of a base, and if there are bounds on the size, perhaps if there is an efficient way to find a minimum (smallest) base. For instance, it can be easily seen that for $G = \Sym(n)$ the symmetric group on $n$ elements $\Omega = \{1,\dotsc,n\}$, a minimum base has size $b(G) = n-1$. However, \added{4}{a natural permutation representation of a dihedral group $G = D_{2n}$ (that also acts on $\Omega$) has minimal base size 2}. Since every element in a permutation group is completely determined by its action on a base, a small base can reduce the space required to store the group.

In \cite{blaha1992}, Blaha discusses the question of whether a polynomial time greedy algorithm suggested in \cite{brown1989} always finds a minimum base for a permutation group $G$, to answer the question ``for what $r$ does a permutation group $G$ have a base of size at most $r$?'' He shows that this is not the case, and even when restricted to elementary abelian groups, the problem is NP-hard. \added{2}{However}, Blaha shows that the algorithm presented produces a base of size $O(b(G)\log\log n)$ where $b(G)$ is the size of a minimum base for $G \leq \Sym(n)$, and that in the worst case, this bound, \added{2}{which is asymptotically larger than $b(G)$,} is actually attained. \added{4}{(Note that all logarithms in this \thesis{} are base-2.)}

\added{4}{Another problem that has traditionally attracted much attention was bounding the order of a finite primitive (permutation) group of degree $n$, i.e. a permutation group on $\Omega$ whose natural action is transitive and such that no nontrivial partition of $\Omega$ is preserved by the action \cite{moscatiello_roney-dougal2021}. Since $|G| \leq n^{b(G)}$ (see \autoref{lem:blaha_nonredundant_size}), we can find upper bounds on $|G|$ by finding upper bounds on $b(G)$. In 1889, Bochert proved in \cite{bochert1889} that for a primitive group $G$ of degree $n$ not containing the alternating group $\Alt(n)$, then $b(G) \leq n/2$; \cite{moscatiello_roney-dougal2021} notes that this was one of the first results in this direction.

    In 1984, Liebeck proved in \cite{liebeck1984} that if $G$ is a primitive group of degree $n$ that is not ``large base'' (in the sense of containing a power of an alternating group as a subgroup), then $b(G) < 9\log n$. The result was proven in context improving lower bounds on $\mu(G)$, the minimal degree of $G$, which is the smallest number of points moved by any non-identity element in $G$. This proof utilised the O'Nan-Scott theorem, which classifies primitive groups according to the structures of their \textit{socles}, and also the classification of finite simple groups (which was recently completed at the time), and was able to achieve a much tighter lower bound of $\mu(G) > n/(9 \log n)$ than the best result that was previously available, which was $\mu(G) > (1/2)(\sqrt{n} - 1)$ and due to Babai in \cite{babai1981}.

    In 2021, building on more recent results, Moscatiello and Roney-Dougal proved in \cite{moscatiello_roney-dougal2021} that if $G$ is a primitive group of degree $n$ that is not large base, then either $G$ is the Mathieu group $M_{24}$, or $b(G) \leq \lceil\log n\rceil + 1$; moreover, there are infinitely many groups for which $b(G) > \log n + 1$. The Mathieu group $M_{24}$ is a member of the first family of sporadic simple groups discovered as part of the classification of finite simple groups, which are 26 finite simple groups that do not fit into the three infinite families (cyclic, alternating, Lie type).

    The authors of \cite{moscatiello_roney-dougal2021} then present a question, which asks to identify primitive groups $G$ of degree $n$ that satisfy $b(G) = \log n + 1$. They note that $G$ must be a subgroup of the affine group $\AGL_d(2)$ of affine transformations of $\F_2^d$ for some $d$, and that if $d$ is even then groups such as the split extension $2^d : \Sp(d,2)$ have this property. Upon investigation using a recursive approach for $d \leq 10$, we found that apart from the affine group $\AGL_d(2)$, there were no other primitive groups for odd $d$, and for even $d$, the only other example found was $2^d : \Sp(d,2)$. Consequently, we conjecture in this \thesis{} that these are the only groups satisfying this property. Limitations of this approach include memory and computational time, since the group found for $d = 10$ has order $25\,410\,822\,678\,459\,187\,200 \approx 2.5 \cdot 10^{19}$ and thus has many subgroups.}

The first part of this \thesis{} largely builds up to a discussion of the results about the NP-hardness of the minimum base problem in \cite{blaha1992}. Firstly, a detailed introduction to group-theoretic concepts such as group actions, bases and stabiliser chains is presented. We apply the concepts of bases, transitivity and primitivity to the Rubik's group $G$, a permutation group of degree 48 that describes valid moves for the Rubik's cube, and show that $b(G) = 18$. Necessary concepts from complexity theory are introduced for a detailed discussion of \cite{blaha1992} that expands on Blaha's proof \added{4}{that finding a minimum base is NP-hard (even for groups with bounded orbits), and we summarise the results on sharp bounds for nonredundant and greedy bases. Afterwards, we review classification results for permutation groups, including the classification of finite simple groups, semidirect and wreath products, affine groups, and the influential O'Nan-Scott theorem for primitive groups. Finally, we discuss improvements in bounds on $b(G)$ for non-large-base primitive groups $G$ as found by Liebeck in \cite{liebeck1984} and Moscatiello and Roney-Dougal in \cite{moscatiello_roney-dougal2021}, and investigate the question posed on primitive groups of degree $n$ that satisfy $b(G) = \log n + 1$. Future directions include more investigation into the question from \cite{moscatiello_roney-dougal2021} and the conjecture presented in this \thesis{}, collecting various known results and bounds for base sizes of large-base primitive groups and imprimitive permutation groups, and identifying special classes of examples where specific bounds or results can be newly found in relation to the NP-hardness of the minimum base problem and analysis of the greedy base algorithm.}

% \begin{itemize}
%     \item What is the problem
%     \item Overview of CGT
%     \item Why are minimum bases desirable
%     \item Tradeoffs with size of transversals?
%     \item Blaha's paper
%     \item Recent advances; what is known
%     \item List notation (where?)
%     \item Emphasise that chapters 2 \& 3 are new?
% \end{itemize}